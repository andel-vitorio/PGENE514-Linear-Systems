\chapter{Conclusão}

Os algoritmos de Gram-Schmidt, tanto na versão clássica quanto na versão modificada, são amplamente utilizados para a ortogonalização de vetores e para a fatoração QR de matrizes. A fatoração QR é uma ferramenta essencial em álgebra linear, com diversas aplicações, incluindo a solução de sistemas lineares, cálculo de determinantes e análise de estabilidade de sistemas dinâmicos. 

A versão clássica do algoritmo realiza a ortogonalização através de projeções sucessivas, enquanto a versão modificada oferece uma abordagem mais eficiente, realizando as projeções de forma direta e evitando a subtração repetida de componentes ortogonais. Sendo a versão modificada, a que permite a decomposição de uma matriz \( A \) em duas matrizes: \( Q \), uma matriz ortonormal, e \( R \), uma matriz triangular superior, o que simplifica a resolução de sistemas lineares e facilita a análise da matriz original. 

Embora ambos os métodos sejam eficazes, a versão modificada é mais estável numericamente, especialmente em casos com vetores mal condicionados. A fatoração QR, derivada desses algoritmos, é crucial para diversas aplicações em ciência e engenharia, demonstrando a importância dos métodos de ortogonalização na prática. Ambos os algoritmos oferecem soluções robustas e são amplamente utilizados em diversos contextos, com a versão modificada oferecendo vantagens adicionais em termos de precisão e desempenho computacional.