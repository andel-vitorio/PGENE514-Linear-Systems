\chapter{Algoritmos Implementados}

O código a seguir realiza a compressão de imagens utilizando a linguagem {\it Python} e os seus pacotes \textit{numpy} e \textit{matplotlib}.

\vspace{8pt}
\begin{lstlisting}[language=Python, caption={Implementação do algoritmo de compressão de imagens usando SVD.}, label={cod:a1}]
import numpy as np
import matplotlib.pyplot as plt
from matplotlib.image import imread

def compress_image(image, k):
  """
  Aplica a decomposição SVD e retorna a imagem reconstruída com rank k.
  :param image: Matriz representando a imagem (grayscale ou canal de cor).
  :param k: Número de componentes a serem mantidos.
  :return: Imagem reconstruída com rank k.
  """
  U, S, Vt = np.linalg.svd(image, full_matrices=False)
  S_k = np.diag(S[:k])  # Reduz os valores singulares para os k primeiros
  return U[:, :k] @ S_k @ Vt[:k, :]


# Carregar imagem em escala de cinza
image = imread("image.jpeg")  # Caminho da imagem
if image.ndim == 3:  # Converte para escala de cinza, se necessário
  image = np.mean(image, axis=2)

# Parâmetros de compressão
ranks = [5, 20, 50, 100]  # Níveis de compressão (postos)
fig, axs = plt.subplots(2, 2, figsize=(10, 10))

# Aplicar compressão e exibir resultados
for i, k in enumerate(ranks):
  row, col = divmod(i, 2)
  compressed_image = compress_image(image, k)
  axs[row, col].imshow(compressed_image, cmap='gray')
  axs[row, col].set_title(f"Posto {k}")
  axs[row, col].axis('off')

plt.tight_layout()
plt.show()
\end{lstlisting}