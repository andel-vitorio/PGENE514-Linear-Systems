\chapter{Conclusão}

A SVD é uma ferramenta fundamental em álgebra linear, amplamente utilizada para analisar e manipular matrizes em diversas aplicações práticas. Ao fatorar uma matriz \( A \) em três componentes — \( U \), \( \Sigma \) e \( V^\top \) —, a SVD permite interpretar as propriedades intrínsecas da matriz, como sua dimensão efetiva, dependência linear entre linhas e colunas e comportamento em transformações lineares. 

O exemplo da compressão de imagens ilustra de forma prática a capacidade da SVD de capturar as informações mais relevantes de uma matriz ao reter apenas os maiores valores singulares. No entanto, suas aplicações vão muito além, abrangendo desde a solução de sistemas lineares e aproximações de matrizes até a análise de dados em estatística e aprendizado de máquina. A capacidade de reduzir matrizes complexas a formas simplificadas e interpretáveis torna a SVD uma ferramenta indispensável em áreas como processamento de sinais e reconhecimento de padrões.

Além disso, a SVD destaca-se por sua robustez, permitindo lidar com matrizes retangulares e mal-condicionadas, oferecendo uma base teórica sólida para resolver problemas de otimização e análise de estabilidade. Assim, a decomposição em valores singulares não apenas exemplifica a versatilidade da álgebra linear, mas também consolida seu papel essencial na ciência e na engenharia modernas.