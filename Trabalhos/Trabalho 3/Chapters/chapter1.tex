\chapter{Introdução}

A decomposição em valores singulares (SVD, do inglês \textit{Singular Value Decomposition}) é uma das técnicas mais poderosas e versáteis em álgebra linear, com ampla aplicação em diversas áreas da matemática, ciência e engenharia. Ela proporciona uma maneira eficiente de analisar e entender as propriedades de matrizes, seja em problemas de análise de dados, processamento de sinais, ou modelagem matemática. Ao decompor uma matriz em três componentes — uma matriz ortogonal \( U \), uma matriz diagonal \( \Sigma \), e uma matriz ortogonal transposta \( V^\top \) — a SVD revela informações cruciais sobre a estrutura da matriz, como sua dimensionalidade, dependências lineares e comportamento sob transformações lineares.

Além de sua importância teórica, a SVD possui um grande número de aplicações práticas. Ela é essencial para a resolução de sistemas lineares, a compressão de dados, a aproximação de matrizes, e é amplamente utilizada em áreas como estatística e aprendizado de máquina. O poder da SVD reside na sua capacidade de simplificar problemas complexos, fornecendo representações compactas e mais compreensíveis de grandes volumes de dados.

Este trabalho explora a teoria e as aplicações da decomposição em valores singulares, destacando sua importância para a compreensão e manipulação de matrizes em uma variedade de contextos práticos. Através de exemplos, como a compressão de imagens, ilustramos o impacto da SVD na simplificação e otimização de processos computacionais.
