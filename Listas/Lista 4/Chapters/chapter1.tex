\chapter{Resolução da Lista de Exercícios}

% =========================== Questão 01 ===========================

\begin{question}
  Encontre os autovalores e os autovetores das seguintes matrizes:
  \begin{multicols}{3}
    \begin{enumerate}[label=\alph*)]
      \item $\begin{bmatrix}
                1 & -2 \\ -2 & 1
              \end{bmatrix}$
      \item $\begin{bmatrix}
                1 & -\frac{2}{3} \\ \frac{1}{2} & \frac{1}{6}
              \end{bmatrix}$
      \item $\begin{bmatrix}
                3 & 1 \\ -1 & 1
              \end{bmatrix}$
      \item $\begin{bmatrix}
                1 & 2 \\ -1 & 1
              \end{bmatrix}$
      \item $\begin{bmatrix}
                3 & -1 & 0 \\ -1 & 2 & -1 \\ 0 & -1 & 3
              \end{bmatrix}$
      \item $\begin{bmatrix}
                -1 & -1 & 4 \\ 1 & 3 & -2 \\ 1 & 1 & -1
              \end{bmatrix}$
      \item $\begin{bmatrix}
                1 & -3 & 11 \\ 2 & -6 & 16 \\ 1 & -3 & 7
              \end{bmatrix}$
      \item $\begin{bmatrix}
                2 & -1 & -1 \\ -2 & 1 & 1 \\ 1 & 0 & 1
              \end{bmatrix}$
      \item $\begin{bmatrix}
                -4 & 4 & 2 \\ 3 & 4 & -1 \\ -3 & -2 & 3
              \end{bmatrix}$
      \item $\begin{bmatrix}
                3 & 4 & 0 & 0 \\ 4 & 3 & 0 & 0 \\ 0 & 0 & 1 & 3 \\ 0 & 0 & 4 & 5
              \end{bmatrix}$
      \item $\begin{bmatrix}
                4 & 0 & 0 & 0 \\ 1 & 3 & 0 & 0 \\ -1 & 1 & 2 & 0 \\ 1 & -1  & 1 & 1
              \end{bmatrix}$
    \end{enumerate}
  \end{multicols}
\end{question}

% ========================= Resolução 01 =========================
\begin{resolution}

  {\bf (a)} \; Os autovalores \(\lambda\) são encontrados resolvendo o determinante da matriz característica \( A - \lambda I = 0 \), onde \( I \) é a matriz identidade.
  \[
    A - \lambda I =
    \begin{bmatrix}
      1 - \lambda & -2          \\
      -2          & 1 - \lambda
    \end{bmatrix}.
  \]
  O determinante é dado por:
  \[
    \det(A - \lambda I) = \det\begin{bmatrix}
      1 - \lambda & -2          \\
      -2          & 1 - \lambda
    \end{bmatrix}.
  \]
  Calculando o determinante:
  \[
    \det(A - \lambda I) = (1 - \lambda)(1 - \lambda) - (-2)(-2),
  \]
  \[
    \det(A - \lambda I) = (1 - \lambda)^2 - 4,
  \]
  \[
    \det(A - \lambda I) = 1 - 2\lambda + \lambda^2 - 4,
  \]
  \[
    \det(A - \lambda I) = \lambda^2 - 2\lambda - 3.
  \]
  Agora resolve-se a equação característica:
  \[
    \lambda^2 - 2\lambda - 3 = 0.
  \]
  Fatorando:
  \[
    \lambda^2 - 2\lambda - 3 = (\lambda - 3)(\lambda + 1) = 0.
  \]
  Portanto, os autovalores são:
  \[
    \lambda_1 = 3, \quad \lambda_2 = -1.
  \]
  Os autovetores são encontrados resolvendo \((A - \lambda I)v = 0\) para cada \(\lambda\). Para \(\lambda_1 = 3\):
  \[
    A - 3I = \begin{bmatrix}
      1 - 3 & -2    \\
      -2    & 1 - 3
    \end{bmatrix}
    = \begin{bmatrix}
      -2 & -2 \\
      -2 & -2
    \end{bmatrix}.
  \]

  Resolvendo \((A - 3I)v = 0\):
  \[
    \begin{bmatrix}
      -2 & -2 \\
      -2 & -2
    \end{bmatrix}
    \begin{bmatrix}
      v_1 \\ v_2
    \end{bmatrix}
    = \begin{bmatrix}
      0 \\ 0
    \end{bmatrix}.
  \]

  A equação reduzida é:
  \[
    -2v_1 - 2v_2 = 0 \quad \Rightarrow \quad v_1 + v_2 = 0.
  \]

  Escolhendo \(v_1 = 1\), então \(v_2 = -1\). O autovetor associado a \(\lambda_1 = 3\) é:
  \[
    v_1 = \begin{bmatrix}
      1 \\ -1
    \end{bmatrix}.
  \]

  Para \(\lambda_2 = -1\):
  \[
    A - (-1)I = \begin{bmatrix}
      1 + 1 & -2    \\
      -2    & 1 + 1
    \end{bmatrix}
    = \begin{bmatrix}
      2  & -2 \\
      -2 & 2
    \end{bmatrix}.
  \]

  Resolvendo \((A + I)v = 0\):
  \[
    \begin{bmatrix}
      2  & -2 \\
      -2 & 2
    \end{bmatrix}
    \begin{bmatrix}
      v_1 \\ v_2
    \end{bmatrix}
    = \begin{bmatrix}
      0 \\ 0
    \end{bmatrix}.
  \]

  A equação reduzida é:
  \[
    2v_1 - 2v_2 = 0 \quad \Rightarrow \quad v_1 = v_2.
  \]

  Escolhendo \(v_1 = 1\), então \(v_2 = 1\). O autovetor associado a \(\lambda_2 = -1\) é:
  \[
    v_2 = \begin{bmatrix}
      1 \\ 1
    \end{bmatrix}.
  \]

  {\bf (b)} \;
  Resolve-se a equação característica \( \det(A - \lambda I) = 0 \), que resulta em:
  \[
    6\lambda^2 - 7\lambda + 3 = 0.
  \]
  Como o discriminante é negativo (\(\Delta = -23\)), os autovalores são complexos:
  \[
    \lambda_1 = \frac{7}{12} + i\frac{\sqrt{23}}{12}, \quad \lambda_2 = \frac{7}{12} - i\frac{\sqrt{23}}{12}.
  \]
  Para cada \(\lambda\), resolve-se o sistema \((A - \lambda I)v = 0\). Os autovetores associados são:
  \[
    v_1 = \begin{bmatrix} \frac{5}{6} + i\frac{\sqrt{23}}{6} \\ 1\end{bmatrix}, \quad
    v_2 = \begin{bmatrix} \frac{5}{6} - i\frac{\sqrt{23}}{6} \\ 1 \end{bmatrix}.
  \]

  {\bf (c)} \;
  Resolve-se a equação característica \( \det(A - \lambda I) = 0 \), que resulta em:
  \[
    \lambda^2-4\lambda+4 = 0.
  \]
  Logo, os autovalores são:
  \[
    \lambda_{1,2} = 2.
  \]
  Para cada \(\lambda\), resolve-se o sistema \((A - \lambda I)v = 0\), ou seja,
  $$
    \begin{bmatrix}
      3 - \lambda & 1 \\ -1 & 1 - \lambda
    \end{bmatrix} \begin{bmatrix}
      v_1 \\ v_2
    \end{bmatrix} = \begin{bmatrix}
      0 \\ 0
    \end{bmatrix} \Rightarrow
    \begin{cases}
      (3 - \lambda) v_1 + v_2 = 0 \\
      -v_1 + (1 - \lambda) v_2 = 0
    \end{cases} = \begin{cases}
      v_1 + v_2 = 0 \\
      -v_1 - v_2 = 0
    \end{cases}
  $$
  Assim, os autovetores associados são:
  \[
    v_{1,2} = \begin{bmatrix} 1 \\ -1\end{bmatrix}
  \]

  {\bf (d)} \;
  Resolve-se a equação característica \( \det(A - \lambda I) = 0 \), que resulta em:
  \[
    \lambda^2 - 2\lambda + 3 = 0.
  \]
  Como o discriminante é negativo (\(\Delta = -8\)), os autovalores são complexos:
  \[
    \lambda_1 = 1 + i\frac{\sqrt{2}}{2}, \quad \lambda_2 = \frac{7}{12} - i\frac{\sqrt{2}}{2}.
  \]
  Para cada \(\lambda\), resolve-se o sistema \((A - \lambda I)v = 0\), ou seja,
  $$
    \begin{bmatrix}
      1 - \lambda & 2 \\ -1 & 1 - \lambda
    \end{bmatrix} \begin{bmatrix}
      v_1 \\ v_2
    \end{bmatrix} = \begin{bmatrix}
      0 \\ 0
    \end{bmatrix} \Rightarrow
    \begin{cases}
      (1 - \lambda) v_1 + v_2 = 0 \\
      -v_1 + (1 - \lambda) v_2 = 0
    \end{cases}.
  $$
  Assim, os autovetores associados são:
  \[
    v_1 = \begin{bmatrix} -i\sqrt{2} \\ 1\end{bmatrix}, \quad
    v_2 = \begin{bmatrix} i \sqrt{2} \\ 1 \end{bmatrix}.
  \]

  {\bf (e)} \;
  Resolve-se a equação característica \( \det(A - \lambda I) = 0 \), que resulta em:
  \[
    -\lambda^3+8\lambda^2-19\lambda+12 = 0
  \]
  Logo, os autovalores são:
  \[
    \lambda_1 = 1, \quad \lambda_2 = 3, \quad \lambda_3 = 4.
  \]
  Para cada \(\lambda\), resolve-se o sistema \((A - \lambda I)v = 0\), ou seja,
  $$
    \begin{bmatrix}
      3 - \lambda & -1 & 0 \\ -1 & 2 - \lambda & -1 \\ 0 & -1 & 3 - \lambda
    \end{bmatrix} \begin{bmatrix}
      v_1 \\ v_2 \\ v_3
    \end{bmatrix} = \begin{bmatrix}
      0 \\ 0 \\ 0
    \end{bmatrix} \Rightarrow
    \begin{cases}
      (3 - \lambda) v_1 - v_2 = 0        \\
      -v_1 + (2 - \lambda) v_2 - v_3 = 0 \\
      -v_2 + (3 - \lambda) v_3 = 0
    \end{cases}.
  $$
  Assim, os autovetores associados são:
  \[
    v_1 = \begin{bmatrix} 1 \\ 2 \\ 1\end{bmatrix}, \quad
    v_2 = \begin{bmatrix} -1 \\ 0 \\ 1\end{bmatrix}, \quad
    v_3 = \begin{bmatrix} 1 \\ -1 \\ 1 \end{bmatrix}.
  \]

  {\bf (f)} \;
  Resolve-se a equação característica \( \det(A - \lambda I) = 0 \), que resulta em:
  \[
    -\lambda^3+\lambda^2+6\lambda-6=0
  \]
  Logo, os autovalores são:
  \[
    \lambda_1=1,\:\lambda_2=-\sqrt{6},\:\lambda_3=\sqrt{6}
  \]
  Para cada \(\lambda\), resolve-se o sistema \((A - \lambda I)v = 0\), ou seja,
  $$
    \begin{bmatrix}
      -1 - \lambda & -1 & 4 \\ 1 & 3 - \lambda & -2 \\ 1 & 1 & -1 - \lambda
    \end{bmatrix} \begin{bmatrix}
      v_1 \\ v_2 \\ v_3
    \end{bmatrix} = \begin{bmatrix}
      0 \\ 0 \\ 0
    \end{bmatrix} \Rightarrow
    \begin{cases}
      (-1 - \lambda) v_1 - v_2 + 4v_3 = 0 \\
      -v_1 + (3 - \lambda) v_2 - 2v_3 = 0 \\
      v_1 + v_2 + (-1 - \lambda) v_3 = 0
    \end{cases}.
  $$
  Assim, os autovetores associados são:
  \[
    v_1 = \begin{bmatrix}2\\ 0\\ 1\end{bmatrix},\:
    v_2 = \begin{bmatrix}-2-\sqrt{6}\\ 4-\sqrt{6}\\ 2\end{bmatrix},\:
    v_3 = \begin{bmatrix}-2+\sqrt{6}\\ 4+\sqrt{6}\\ 2\end{bmatrix}.
  \]

  {\bf (g)} \;
  Resolve-se a equação característica \( \det(A - \lambda I) = 0 \), que resulta em:
  \[
    -\lambda^3+12\lambda=0
  \]
  Logo, os autovalores são:
  \[
    \lambda_1=0,\:\lambda_2=-2\sqrt{3},\:\lambda_3=2\sqrt{3}.
  \]
  Para cada \(\lambda\), resolve-se o sistema \((A - \lambda I)v = 0\), ou seja,
  $$
    \begin{bmatrix}
      -1 - \lambda & 3 & 11 \\ 2 & -6 - \lambda & 16 \\ 1 & -3 & 7 - \lambda
    \end{bmatrix} \begin{bmatrix}
      v_1 \\ v_2 \\ v_3
    \end{bmatrix} = \begin{bmatrix}
      0 \\ 0 \\ 0
    \end{bmatrix} \Rightarrow
    \begin{cases}
      (-1 - \lambda) v_1 + 3 v_2 + 11 v_3 = 0 \\
      2 v_1 + (-6 - \lambda) v_2 + 16 v_3 = 0 \\
      v_1 - 3 v_2 + (7 - \lambda) v_3 = 0
    \end{cases}.
  $$
  Assim, os autovetores associados são:
  \[
    v_1 = \begin{bmatrix}3\\ 1\\ 0\end{bmatrix},\:
    v_2 =\begin{bmatrix}-3-9\sqrt{3}\\ 6-\sqrt{3}\\ 3\end{bmatrix},\:
    v_3 =\begin{bmatrix}-3+9\sqrt{3}\\ 6+\sqrt{3}\\ 3\end{bmatrix}.
  \]

  {\bf (h)} \;
  Resolve-se a equação característica \( \det(A - \lambda I) = 0 \), que resulta em:
  \[
    -\lambda^3+4\lambda^2-4\lambda=0
  \]
  Logo, os autovalores são:
  \[
    \lambda_1=0,\:\lambda_{2, 3}=2.
  \]
  Para cada \(\lambda\), resolve-se o sistema \((A - \lambda I)v = 0\). Os autovetores associados são:
  \[
    v_1 = \begin{bmatrix}-1\\ -3\\ 1\end{bmatrix},\:
    v_2 = \begin{bmatrix}1\\ -1\\ 1\end{bmatrix}.
  \]

  {\bf (i)} \;
  Resolve-se a equação característica \( \det(A - \lambda I) = 0 \), que resulta em:
  \[
    -\lambda^3+3\lambda^2+24\lambda-52=0
  \]
  Logo, os autovalores são:
  \[
    \lambda_1=2,\:\lambda_2=\frac{1+\sqrt{105}}{2},\:\lambda_3=\frac{1-\sqrt{105}}{2}
  \]
  Para cada \(\lambda\), resolve-se o sistema \((A - \lambda I)v = 0\). Os autovetores associados são:
  \[
    v_1 = \begin{bmatrix}1\\ 0\\ 3\end{bmatrix},\:
    v_2 = \begin{bmatrix}9-\sqrt{105}\\ -6\\ 6\end{bmatrix},\:
    v_3 = \begin{bmatrix}9+\sqrt{105}\\ -6\\ 6\end{bmatrix}.
  \]

  {\bf (j)} \;
  Resolve-se a equação característica \( \det(A - \lambda I) = 0 \), que resulta em:
  \[
    \lambda^4-12\lambda^3+22\lambda^2+84\lambda+49=0
  \]
  Logo, os autovalores são:
  \[
    \lambda_{1, 2}=-1,\:\lambda_{3, 4}=7.
  \]
  Para cada \(\lambda\), resolve-se o sistema \((A - \lambda I)v = 0\). Os autovetores associados são:
  \[
    v_1 = \begin{pmatrix}-1\\ 1\\ 0\\ 0\end{pmatrix},\:
    v_2 = \begin{pmatrix}0\\ 0\\ -\frac{3}{2}\\ 1\end{pmatrix},\:
    v_3 = \begin{pmatrix}1\\ 1\\ 0\\ 0\end{pmatrix},\:
    v_4 = \begin{pmatrix}0\\ 0\\ \frac{1}{2}\\ 1\end{pmatrix}.
  \]

  {\bf (k)} \;
  Resolve-se a equação característica \( \det(A - \lambda I) = 0 \), que resulta em:
  \[
    \left(-\lambda+4\right)\left(-\lambda+3\right)\left(-\lambda+2\right)\left(-\lambda+1\right)=0
  \]
  Logo, os autovalores são:
  \[
    \lambda_1=4,\:\lambda_2=3,\:\lambda_3=2,\:\lambda_4=1.
  \]
  Para cada \(\lambda\), resolve-se o sistema \((A - \lambda I)v = 0\). Os autovetores associados são:
  \[
    v_1 = \begin{pmatrix}1\\ 1\\ 0\\ 0\end{pmatrix},\:
    v_2 = \begin{pmatrix}0\\ 1\\ 1\\ 0\end{pmatrix},\:
    v_3 = \begin{pmatrix}0\\ 0\\ 1\\ 1\end{pmatrix},\:
    v_4 = \begin{pmatrix}0\\ 0\\ 0\\ 1\end{pmatrix}.
  \]
\end{resolution}

% =========================== Questão 02 ===========================
\begin{question}
  \vspace{-24pt}
  \begin{enumerate}[label=\alph*)]
    \item Calcule os autovalores e os autovetores correspondentes de $A = \begin{bmatrix}
              1 & 4 & 4 \\ 3 & -1 & 0 \\ 0 & 2 & 3
            \end{bmatrix}$

    \item Calcule o traço de $A$ e verifique que é igual a soma dos autovalores.
    \item Encontre o determinante de $A$ e verifique que é igual ao produto dos autovalores.
  \end{enumerate}
\end{question}

% ========================= Resolução 02 =========================
\begin{resolution}
  Primeiramente, inicia-se pelo cálculo dos autovalores e autovetores da matriz \( A = \begin{bmatrix} 1 & 4 & 4 \\ 3 & -1 & 0 \\ 0 & 2 & 3 \end{bmatrix} \). Os autovalores são encontrados resolvendo o determinante da matriz característica \( A - \lambda I \), ou seja, \(\det(A - \lambda I) = 0\). A matriz característica é dada por
  \[
    A - \lambda I = \begin{bmatrix}
      1 - \lambda & 4            & 4           \\
      3           & -1 - \lambda & 0           \\
      0           & 2            & 3 - \lambda
    \end{bmatrix}.
  \]
  O determinante dessa matriz é
  \[
    \det(A - \lambda I) = \begin{vmatrix}
      1 - \lambda & 4            & 4           \\
      3           & -1 - \lambda & 0           \\
      0           & 2            & 3 - \lambda
    \end{vmatrix}
    = -\lambda^3+3\lambda^2+13\lambda-15.
  \]
  Os autovalores de \( A \) são as raízes da equação cúbica \(-\lambda^3+3\lambda^2+13\lambda-15=0\). Aplicando métodos de fatoração, obtém-se que os autovalores são \( \lambda_1 = 1 \), \( \lambda_2 = -3 \), \( \lambda_3 = 5 \).

  Para os autovetores, resolve-se \( (A - \lambda I)v = 0 \) para cada autovalor. Para \( \lambda_1 = 1 \), tem-se a matriz
  \[
    A - I = \begin{bmatrix}
      0 & 4  & 4 \\
      3 & -2 & 0 \\
      0 & 2  & 2
    \end{bmatrix},
  \]
  e a solução do sistema linear correspondente fornece o autovetor \( v_1 = \begin{bmatrix} -2 \\ -3 \\ 3 \end{bmatrix} \). Analogamente, para \( \lambda_2 = -3 \), a matriz
  \[
    A + 3I = \begin{bmatrix}
      4 & 4 & 4 \\
      3 & 2 & 0 \\
      0 & 2 & 6
    \end{bmatrix}
  \]
  leva ao autovetor \( v_2 = \begin{bmatrix} 2 \\ -3 \\ 1 \end{bmatrix} \). Por fim, para \( \lambda_3 = 5 \), resolve-se
  \[
    A - 5I = \begin{bmatrix}
      -4 & 4  & 4  \\
      3  & -6 & 0  \\
      0  & 2  & -2
    \end{bmatrix},
  \]
  resultando no autovetor \( v_3 = \begin{bmatrix} 2 \\ 1 \\ 1 \end{bmatrix} \).

  O traço de \( A \) é a soma dos elementos da diagonal principal, ou seja, \( 1 + (-1) + 3 = 3 \). Nota-se que isso é igual à soma dos autovalores \( 1 + (-3) + 5 = 3 \), confirmando a relação. O determinante de \( A \), calculado diretamente, é \( -15 \), o que corresponde ao produto dos autovalores \( 1 \cdot (-3) \cdot (5) = -15 \), confirmando a propriedade.
\end{resolution}

% =========================== Questão 03 ===========================
\begin{question}
  Encontre os autovalores e a base de cada autoespaço das seguinte matrizes:
  \begin{multicols}{3}
    \begin{enumerate}[label=\alph*)]
      \item $\begin{bmatrix}
                4 & -4 \\ 1 & 0
              \end{bmatrix}$
      \item $\begin{bmatrix}
                6 & -8 \\ 4  & -6
              \end{bmatrix}$
      \item $\begin{bmatrix}
                3 & -2 \\ 4 & -1
              \end{bmatrix}$
      \item $\begin{bmatrix}
                i & -1 \\1 & i
              \end{bmatrix}$
      \item $\begin{bmatrix}
                4 & -1 & -1 \\ 0 & 3 & 0 \\ 1 & -1 & 2
              \end{bmatrix}$
      \item $\begin{bmatrix}
                -6 & 0 & -8 \\ -4 & 2 & -4 \\ 4 & 0 & 6
              \end{bmatrix}$
      \item $\begin{bmatrix}
                -2 & 1 & -1 \\ 5 & -3 & 6 \\ 5 & -1 & 4
              \end{bmatrix}$
      \item $\begin{bmatrix}
                1 & 0 & 0 & 0 \\ 0 & 1 & 0 & 0 \\ -1 & 1 & -1 & 0 \\ 1 & 0 & -1 & 0
              \end{bmatrix}$
      \item $\begin{bmatrix}
                -1 & 0 & 1 & 2 \\ 0 & 1 & 0 & 1 \\ -1 & -4 & 1 & -2 \\ 0 & 1 & 0 & 1
              \end{bmatrix}$
    \end{enumerate}
  \end{multicols}
\end{question}

% ========================= Resolução 03 =========================
\begin{resolution}
  {\bf (a)} Para a matriz \( A = \begin{bmatrix} 4 & -4 \\ 1 & 0 \end{bmatrix} \), inicia-se determinando os autovalores resolvendo o determinante da matriz característica \( A - \lambda I \). Tem-se:
  \[
    A - \lambda I = \begin{bmatrix} 4 - \lambda & -4 \\ 1 & -\lambda \end{bmatrix}.
  \]
  O determinante dessa matriz é dado por:
  \[
    \det(A - \lambda I) = (4 - \lambda)(-\lambda) - (-4)(1).
  \]
  Expandindo os termos:
  \[
    \det(A - \lambda I) = -4\lambda + \lambda^2 + 4.
  \]
  Fatorando a equação quadrática:
  \[
    \lambda^2 - 4\lambda + 4 = (\lambda - 2)^2.
  \]
  Logo, o autovalor de \( A \) é \(\lambda = 2\), com multiplicidade algébrica 2. Para determinar a base do autoespaço associado, resolve-se o sistema \( (A - \lambda I)v = 0 \). Substituindo \(\lambda = 2\), obtém-se:
  \[
    A - 2I = \begin{bmatrix} 2 & -4 \\ 1 & -2 \end{bmatrix}.
  \]
  A forma escalonada da matriz é:
  \[
    \begin{bmatrix} 1 & -2 \\ 0 & 0 \end{bmatrix}.
  \]
  O sistema correspondente é:
  \[
    x_1 - 2x_2 = 0.
  \]
  Logo, \( x_1 = 2x_2 \), e a solução geral é:
  \[
    v = x_2 \begin{bmatrix} 2 \\ 1 \end{bmatrix}, \quad x_2 \in \mathbb{R}.
  \]
  Portanto, a base do autoespaço associado a \(\lambda = 2\) é:
  \[
    \mathcal{B} = \left\{ \begin{bmatrix} 2 \\ 1 \end{bmatrix} \right\}.
  \]

  Segue a solução resumida para as matrizes fornecidas, com cálculo dos autovalores e as bases dos autoespaços associados:

  {\bf (b)} Para a matriz \( A = \begin{bmatrix} 6 & -8 \\ 4 & -6 \end{bmatrix} \), temos que o determinante de \( A - \lambda I \):
  \[
    \det(A - \lambda I) = \begin{vmatrix} 6 - \lambda & -8 \\ 4 & -6 - \lambda \end{vmatrix} = \lambda^2 - 0\lambda - 4 = (\lambda - 2)(\lambda + 2).
  \]
  Logo, os autovalores são \( \lambda_1 = 2, \lambda_2 = -2 \), os autoespaços:
  Para \( \lambda_1 = 2 \):
  $$ A - 2I = \begin{bmatrix} 4 & -8 \\ 4 & -8 \end{bmatrix},$$ base \( \mathcal{B}_1 = \left\{ \begin{bmatrix} 2 \\ 1 \end{bmatrix} \right\} \).
  Para \( \lambda_2 = -2 \):
  $$ A + 2I = \begin{bmatrix} 8 & -8 \\ 4 & -4 \end{bmatrix},$$ base \( \mathcal{B}_2 = \left\{ \begin{bmatrix} 1 \\ 1 \end{bmatrix} \right\} \).

  \vspace{12pt}
  {\bf (c)} Para a matriz \( A = \begin{bmatrix} 3 & -2 \\ 4 & -1 \end{bmatrix} \), temos:
  $$ \det(A - \lambda I) = (3 - \lambda)(-1  - \lambda) + 8 = 0 .$$
  Logo, os autovalores são \( \lambda_1 = 1 + 2i, \lambda_2 = 1 - 2i \). Assim,
  para \( \lambda_1 = 1 + 2i\), a base é \( \mathcal{B}_1 = \left\{ \begin{bmatrix} 1 + i \\ 2 \end{bmatrix} \right\} \); e
  para \( \lambda_2 = 1 - 2i \), a base \( \mathcal{B}_2 = \left\{ \begin{bmatrix} 1 - i\\ 2 \end{bmatrix} \right\} \).

  \vspace{12pt}
  {\bf (d)} Para a matriz \( A = \begin{bmatrix} i & -1 \\ 1 & i \end{bmatrix} \), temos:
  $$\det(A - \lambda I) = (\lambda - i)^2 + 1 = 0.$$
  Os autovalores são \( \lambda_1 = 0, \lambda_2 = 2i \). Logo, para \( \lambda_1 = 0\), a base é \( \mathcal{B}_1 = \left\{ \begin{bmatrix} 1 \\ i \end{bmatrix} \right\} \); e para \( \lambda_2 = i - 1 \), a base é \( \mathcal{B}_2 = \left\{ \begin{bmatrix} 1 \\ -i \end{bmatrix} \right\} \).

  \vspace{12pt}
  {\bf (e)} Para a matriz \( A = \begin{bmatrix} 4 & -1 & -1 \\ 0 & 3 & 0 \\ 1 & -1 & 2 \end{bmatrix} \), os autovalores são \( \lambda_{1, 2, 3} = 3 \) (multiplicidade algébrica 3). Dessa forma, os autovetores, para \( \lambda_{1, 2, 3} = 3 \). são obtidos pela base \( \mathcal{B} = \left\{ \begin{bmatrix} 1 \\ 1 \\ 0 \end{bmatrix}, \begin{bmatrix} 1 \\ 0 \\ 1 \end{bmatrix} \right\} \).


  \vspace{12pt}
  {\bf (f)} Para a matriz \( A = \begin{bmatrix} -6 & 0 & -8 \\ -4 & 2 & -4 \\ 4 & 0 & 6 \end{bmatrix} \), os autovalores são \( \lambda_{1,2} = 2, \lambda_3 = -2 \). Logo,
  para \( \lambda_{1,2} = 2 \), a base é \( \mathcal{B}_1 = \left\{ \begin{bmatrix} 0 \\ 1 \\ 0 \end{bmatrix}, \begin{bmatrix} -1 \\ 0 \\ 1 \end{bmatrix} \right\} \).
  Para \( \lambda_3 = -2 \), a base é \( \mathcal{B}_2 = \left\{ \begin{bmatrix} -2 \\ -1 \\ 1 \end{bmatrix} \right\} \).

  \vspace{12pt}
  {\bf (g)} Para a matriz \( A = \begin{bmatrix} -2 & 1 & -1 \\ 5 & -3 & 6 \\ 5 & -1 & 4 \end{bmatrix} \), os autovalores são \( \lambda_{1,2} = -2, \lambda_3 = 3\).  Para cada autovalor, as bases são \( \mathcal{B}_1 = \left\{ \begin{bmatrix} -1 \\ 1 \\ 1 \end{bmatrix} \right\} \) e \( \mathcal{B}_2 = \left\{ \begin{bmatrix} 0 \\ 1 \\ 1 \end{bmatrix} \right\} \).

  \vspace{12pt}
  {\bf (h)} Para a matriz \( A = \begin{bmatrix} 1 & 0 & 0 & 0 \\ 0 & 1 & 0 & 0 \\ -1 & 1 & -1 & 0 \\ 1 & 0 & -1 & 0 \end{bmatrix} \), os autovalores são \( \lambda_{1} = 0, \lambda_{2, 3} = 1, \lambda_4 = -1\).  Para cada autovalor, as bases são \( \mathcal{B}_1 = \left\{ \begin{bmatrix} 0 \\ 0 \\ 0 \\ 1 \end{bmatrix} \right\} \), \( \mathcal{B}_2 = \left\{ \begin{bmatrix} 1 \\ 3 \\ 1 \\ 0 \end{bmatrix}, \begin{bmatrix} 1 \\ 1 \\ 0 \\ 1 \end{bmatrix} \right\} \) e \( \mathcal{B}_3 = \left\{ \begin{bmatrix} 0 \\ 0 \\ 1 \\ 1 \end{bmatrix} \right\} \).

  \vspace{12pt}
  {\bf (i)} Para a matriz \( A = \begin{bmatrix} -1 & 0 & 1 & 2 \\ 0 & 1 & 0 & 1 \\ -1 & -4 & 1 & -2 \\ 0 & 1 & 0 & 1 \end{bmatrix} \), os autovalores são \( \lambda_{1, 2, 3} = 0, \lambda_4 = 2\).  Para cada autovalor, as bases são \( \mathcal{B}_1 = \left\{ \begin{bmatrix} 1 \\ 0 \\ 1 \\ 0 \end{bmatrix} \right\} \) e \( \mathcal{B}_2 = \left\{ \begin{bmatrix} 0 \\ 1 \\ -2 \\ 1 \end{bmatrix}\right\} \).

\end{resolution}

% =========================== Questão 04 ===========================
\begin{question}
  Diagonalize as seguintes matrizes:
  \begin{multicols}{3}
    \begin{enumerate}[label=\alph*)]
      \item $\begin{bmatrix}
                3 & -9 \\ 2 & -6
              \end{bmatrix}$
      \item $\begin{bmatrix}
                5 & -4 \\ 2 & -1
              \end{bmatrix}$
      \item $\begin{bmatrix}
                -4 & -2 \\ 5 & 2
              \end{bmatrix}$
      \item $\begin{bmatrix}
                -2 & 3 & 1 \\ 0 & 1 & -1 \\ 0 & 0 & 3
              \end{bmatrix}$
      \item $\begin{bmatrix}
                8 & 0 & -3 \\ -3 & 0 & -1 \\ 3 & 0 & -2
              \end{bmatrix}$
      \item $\begin{bmatrix}
                3 & 3 & 5 \\ 5 & 6 & 5 \\ -5& -8 & -7
              \end{bmatrix}$
      \item $\begin{bmatrix}
                2 & 5 & 5 \\ 0 & 2 & 0 \\ 0 & -5 & -3
              \end{bmatrix}$
      \item $\begin{bmatrix}
                1 & 0 & -1 & 1 \\ 0 & 2 & -1 & 1 \\ 0 & 0 & -1 & 0 \\ 0 & 0 & 0 & -2
              \end{bmatrix}$
      \item $\begin{bmatrix}
                0 & 0 & 1 & 0 \\ 0 & 0 & 0 & 1 \\ 1 & 0 & 0 & 0 \\ 0 & 1 & 0 & 0
              \end{bmatrix}$
      \item $\begin{bmatrix}
                2 & 1 & -1 & 0 \\ -3 & -2 & 0 & 1 \\ 0 & 0 & 1 & -2 \\ 0 & 0 & 1 & -1
              \end{bmatrix}$
    \end{enumerate}
  \end{multicols}
\end{question}

% ========================= Resolução 04 =========================
\begin{resolution}
  As matrizes foram diagonalizadas usando o algoritmo de decomposição por valores singulares implementado em um trabalho requerido nessa disciplina.

  \vspace{12pt}
  {\bf (a)}\; $\begin{bmatrix}
      3 & -9 \\ 2 & -6
    \end{bmatrix} = \begin{bmatrix}
      -0.8321 & -0.55 \\
      -0.555   & 0.832
    \end{bmatrix} \begin{bmatrix}
      11.4 & 0 \\ 0 & 2.81 \times 10^{-16}
    \end{bmatrix} \begin{bmatrix}
      -0.316 & 0.949 \\ -0.949 & -0.316
    \end{bmatrix}$

  \vspace{12pt}
  {\bf (b)}\; $\begin{bmatrix}
      5 & -4 \\ 2 & -1
    \end{bmatrix} = \begin{bmatrix}
      -0.946 & -0.325 \\
      -0.325 & 0.946
    \end{bmatrix} \begin{bmatrix}
      6.77 & 0 \\ 0 & 0.443
    \end{bmatrix} \begin{bmatrix}
      -0.795 & 0.607 \\ -0.607 & -0.795
    \end{bmatrix}$

  \vspace{12pt}
  {\bf (c)}\; $\begin{bmatrix}
      -4 & -2 \\ 5 & 2
    \end{bmatrix} = \begin{bmatrix}
      -0.639 & -0.770 \\
      -0.770 & 0.639
    \end{bmatrix} \begin{bmatrix}
      6.99 & 0 \\ 0 & 0.286
    \end{bmatrix} \begin{bmatrix}
      0.915 & 0.403 \\ 0.403 & -0.915
    \end{bmatrix}$

  \vspace{12pt}
  {\bf (d)}\; \begin{multline*}\begin{bmatrix}
      -3 & 3 & 1 \\ 0 & 1 & -1 \\ 0 & 0 & 3
    \end{bmatrix} = \begin{bmatrix}
      -0.967 & -0.201 & 0.159 \\
      0.0665 & 0.402  & 0.913 \\
      0.248  & -0.893 & 0.375
    \end{bmatrix} \\ \times \begin{bmatrix}
      4.46 & 0 & 0 \\ 0 & 3.11 & 0 \\ 0 & 0 & 0.649
    \end{bmatrix}  \begin{bmatrix}
      0.650 & 0.665 & 0.368 \\ -0.194 & 0.323 & -0.926 \\ -0.735 & -0.673 & -0.0811
    \end{bmatrix}\end{multline*}

  \vspace{12pt}
  {\bf (e)}\; \begin{multline*}\begin{bmatrix}
      8 & 0 & -3 \\ -3 & 0 & -1 \\ 3 & 0 & -2
    \end{bmatrix} = \begin{bmatrix}
      -0.893 & -0.0926 & 0.440  \\
      0.262  & -0.902  & -0.342 \\
      -0.365 & -0.421  & 0.831
    \end{bmatrix} \\ \times \begin{bmatrix}
      9.56 & 0 & 0 \\ 0 & 2.14 & 0 \\ 0 & 0 & 1.71 \times 10^{-16}
    \end{bmatrix}  \begin{bmatrix}
      -0.944 & 0 & 0.329 \\ 0.329 & 0 & 0.944 \\ 0 & -1 & 2.22 \times 10^{-16}
    \end{bmatrix}\end{multline*}

  \vspace{12pt}
  {\bf (f)}\; \begin{multline*}\begin{bmatrix}
      3 & 3 & 5 \\ 5 & 6 & 5 \\ -5 & -8 & -7
    \end{bmatrix} = \begin{bmatrix}
      -0.393 & 0.914  & 0.101 \\
      -0.569 & -0.327 & 0.754 \\
      0.723  & 0.239  & 0.649
    \end{bmatrix} \\ \times \begin{bmatrix}
      16.2 & 0 & 0 \\ 0 & 1.7 & 0 \\ 0 & 0 & 0.944
    \end{bmatrix}  \begin{bmatrix}
      -0.471 & -0.639 & -0.608 \\ -0.0526 & -0.668 & 0.743 \\ 0.881 & -0.381 & -0.281
    \end{bmatrix}\end{multline*}

  \vspace{12pt}
  {\bf (g)}\; \begin{multline*}\begin{bmatrix}
      2 & 5 & 5 \\ 0 & 2 & 0 \\ 0 & -5 & -3
    \end{bmatrix} = \begin{bmatrix}
      0.775  & 0.577  & 0.256 \\
      0.166  & -0.577 & 0.800 \\
      -0.610 & 0.577  & 0.543
    \end{bmatrix} \\ \times \begin{bmatrix}
      9.36 & 0 & 0 \\ 0 & 2 & 0 \\ 0 & 0 & 0.641
    \end{bmatrix}  \begin{bmatrix}
      0.166 & 0.775 & 0.610 \\ 0.577 & -0.577 & 0.577 \\ 0.8 & 0.256s & -0.543
    \end{bmatrix}\end{multline*}

  \vspace{12pt}
  {\bf (h)}\; \begin{multline*}\begin{bmatrix}
      1 & 0 & -1 & 1 \\ 0 & 2 & -1 & 1 \\ 0 & 0 & -1 & 0 \\ 0 & 0 & 0 & -2
    \end{bmatrix} = \begin{bmatrix}
      0.449  & -0.286 & -0.704 & 0.470  \\
      0.730  & 0.621  & 0.271  & 0.0863 \\
      0.151  & 0.163  & -0.520 & -0.825 \\
      -0.493 & 0.711  & -0.4   & 0.303
    \end{bmatrix} \\ \times \begin{bmatrix}
      2.96 & 0 & 0 & 0 \\ 0 & 1.75 & 0 & 0 \\ 0 & 0 & 1.35 & 0 \\ 0 & 0 & 0 & 0.570
    \end{bmatrix}  \begin{bmatrix}
      0.151  & 0.493 & -0.449 & 0.730 \\ -0.163 & 0.711 & -0.286 & -0.621 \\
      -0.520 & 0.4   & 0.704  & 0.271 \\ 0.825 & 0.303 & 0.470 & 0.0863
    \end{bmatrix}\end{multline*}

  \vspace{12pt}
  {\bf (i)}\; \begin{equation*}\begin{bmatrix}
      0 & 0 & 1 & 0 \\ 0 & 0 & 0 & 1 \\ 1 & 0 & 0 & 0 \\ 0 & 1 & 0 & 0
    \end{bmatrix} = \begin{bmatrix}
      0  & 0  & -1 & 0  \\
      0  & 0  & 0  & -1 \\
      -1 & 0  & 0  & 0  \\
      0  & -1 & 0  & 0
    \end{bmatrix} \begin{bmatrix}
      1 & 0 & 0 & 0 \\ 0 & 1 & 0 & 0 \\ 0 & 0 & 1 & 0 \\ 0 & 0 & 0 & 1
    \end{bmatrix}  \begin{bmatrix}
      -1 & 0 & 0  & 0 \\ 0 & -1 & 0 & 0 \\
      0  & 0 & -1 & 0 \\ 0 & 0 & 0 & -1
    \end{bmatrix}\end{equation*}

  \vspace{12pt}
  {\bf (j)}\; \begin{multline*}\begin{bmatrix}
      2 & 1 & -1 & 0 \\ -3 & -2 & 0 & 1 \\ 0 & 0 & 1 & -2 \\ 0 & 0 & 1 & -1
    \end{bmatrix} = \begin{bmatrix}
      -0.512  & -0.337  & 0.618  & -0.492 \\
      0.853   & -0.0953 & 0.415  & -0.302 \\
      -0.0926 & 0.790   & 0.552  & 0.248  \\
      -0.0362 & 0.503   & -0.376 & -0.778
    \end{bmatrix} \\ \times \begin{bmatrix}
      4.37 & 0 & 0 & 0 \\ 0 & 2.75 & 0 & 0 \\ 0 & 0 & 0.582 & 0 \\ 0 & 0 & 0 & 0.143
    \end{bmatrix}  \begin{bmatrix}
      -0.821  & -0.508 & 0.0879 & 0.246  \\ -0.141 & -0.0532 & 0.592 & -0.792\\
      -0.0169 & -0.365 & -0.758 & -0.540 \\ -0.553 & 0.778 & -0.258 & -0.147
    \end{bmatrix}\end{multline*}

\end{resolution}

% =========================== Questão 05 ===========================
\begin{question}
  Escreva abaixo um matriz real que tenha:
  \begin{enumerate}[label=\alph*)]
    \item autovalores $-1$, $3$ e autovalores correspondentes $\begin{bmatrix}-1 \\ 2\end{bmatrix}, \begin{bmatrix} 1 \\ 1\end{bmatrix}$;
    \item autovalores $0,2, -2$ e autovalores correspondentes $\begin{bmatrix}-1 \\ 1 \\ 0\end{bmatrix}, \begin{bmatrix} 2 \\ -1 \\ 1\end{bmatrix}, \begin{bmatrix} 0 \\ 1 \\ 3\end{bmatrix}$;
    \item um autovalor de  $3$ e um autovalor correspondente $\begin{bmatrix}2\\ -3 \end{bmatrix}, \begin{bmatrix} 1\\ 2\end{bmatrix}$;
    \item autovalores $-1 + 2i$ e autovalor correspondente $\begin{bmatrix}1+ i \\ 3i\end{bmatrix}$;
    \item autovalores $-2$ e autovalor correspondente $\begin{bmatrix}-2 \\ 0 \\ -1\end{bmatrix}$;
  \end{enumerate}
  \vspace{8pt}
\end{question}

% ========================= Resolução 05 =========================
\begin{resolution}

\end{resolution}

% =========================== Questão 06 ===========================
\begin{question}
  Encontre uma base para o complemento ortogonal de cada um dos conjuntos a seguir no espaço de produto interno indicado.
  \begin{itemize}
    \item[(a)] $\{(0,0,0)\} \subset \mathbb{R}^3$;
    \item[(b)] $\{(1,1,1), (2,1,0)\} \subset \mathbb{R}^3$.
  \end{itemize}
\end{question}

% ========================= Resolução 06 =========================
\begin{resolution}

\end{resolution}

% =========================== Questão 07 ===========================
\begin{question}
  Calcule uma base de cada um dos quatro subespaços fundamentais das seguintes matrizes e verifique que elas satisfazem as relações de ortogonalidade:
  \begin{multicols}{2}
    \begin{itemize}
      \item[(a)] $\begin{bmatrix}
                1 & 2 \\
                3 & 6
              \end{bmatrix}$;
      \item[(b)] $\begin{bmatrix}
                2 & 1  & 3 & 1 \\
                4 & -1 & 2 & 3
              \end{bmatrix}$.
    \end{itemize}
  \end{multicols}
  \vspace{8pt}
\end{question}

% ========================= Resolução 07 =========================
\begin{resolution}

\end{resolution}

% =========================== Questão 08 ===========================
\begin{question}
  Calcule a decomposição QR da matriz:
  \[
    A = \begin{bmatrix}
      1  & 3 & 3  \\
      2  & 2 & -2 \\
      -2 & 2 & 1
    \end{bmatrix}.
  \]
\end{question}

% ========================= Resolução 08 =========================
\begin{resolution}

\end{resolution}

% =========================== Questão 09 ===========================
\begin{question}
  Calcule a decomposição em valores singulares (SVD) da matriz $A$:
  \[
    A = \begin{bmatrix}
      1  & 2 & 3 \\
      -1 & 0 & 1 \\
      3  & 2 & 1
    \end{bmatrix}.
  \]
\end{question}

% ========================= Resolução 09 =========================
\begin{resolution}

\end{resolution}

% =========================== Questão 10 ===========================
\begin{question}
  Considere o seguinte conjunto de vetores em $\mathbb{R}^2$ (com o produto interno convencional):
  \[
    S = \left\{ \mathbf{v}_1 =
    \begin{bmatrix}
      3 \\
      1
    \end{bmatrix}, \mathbf{v}_2 =
    \begin{bmatrix}
      2 \\
      2
    \end{bmatrix} \right\}.
  \]
  Agora, realize o processo de Gram-Schmidt para obter um conjunto ortogonal de vetores.
\end{question}

% ========================= Resolução 10 =========================
\begin{resolution}

\end{resolution}

% =========================== Questão 11 ===========================
\begin{question}
  Seja
  \[
    A =
    \begin{bmatrix}
      -1 & 2 & -1 \\
      0  & 1 & 3  \\
      -1 & 2 & 5
    \end{bmatrix}.
  \]
  \begin{itemize}
    \item[a.] Encontre $P(\lambda)$, o polinômio característico de $A$.
    \item[b.] Encontre os autovalores de $A$.
    \item[c.] Mostre que $P(A) = 0$.
  \end{itemize}

\end{question}

% ========================= Resolução 11 =========================
\begin{resolution}

\end{resolution}

% =========================== Questão 12 ===========================
\begin{question}
  Seja a matriz $3 \times 3$ $A$ definida como:
  \[
    A =
    \begin{bmatrix}
      0 & 1 & 0 \\
      1 & 0 & 1 \\
      0 & 1 & 0
    \end{bmatrix}.
  \]
  Calcule $A^{2004}$ e $e^{At}$.
\end{question}

% ========================= Resolução 12 =========================
\begin{resolution}

\end{resolution}

% =========================== Questão 13 ===========================
\begin{question}
  Encontre a representação no espaço de estados para os seguintes sistemas dinâmicos na forma canônica controlável:
  \begin{itemize}
    \item[a.] $\dfrac{d^3 y(t)}{dt^3} + 3 \dfrac{d^2 y(t)}{dt^2} + 4 \dfrac{dy(t)}{dt} + 2y(t) = 7u(t)$.
  \end{itemize}
\end{question}

% ========================= Resolução 13 =========================
\begin{resolution}

\end{resolution}


% =========================== Questão 14 ===========================
\begin{question}
  Determine se as seguintes matrizes são definidas positivas / semi-definidas ou definidas negativas / semi-definidas:
  \begin{multicols}{3}
    \begin{itemize}
      \item[a.] $A = \begin{bmatrix} 3 & 2 \\ 2 & 3 \end{bmatrix}$;
      \item[b.] $B = \begin{bmatrix} 1 & 2 \\ 2 & 4 \end{bmatrix}$;
      \item[c.] $C = \begin{bmatrix} 1 & -2 \\ -2 & -6 \end{bmatrix}$.
    \end{itemize}
  \end{multicols}
  \vspace{8pt}
\end{question}

% ========================= Resolução 14 =========================
\begin{resolution}

\end{resolution}


% =========================== Questão 15 ===========================
\begin{question}
  Considere o sistema no tempo discreto:
  \[
    x(k+1) =
    \begin{bmatrix}
      0 & 1 & 0 & 0 \\
      0 & 0 & 1 & 0 \\
      0 & 0 & 0 & 1 \\
      0 & 0 & 0 & 0
    \end{bmatrix}
    x(k) +
    \begin{bmatrix}
      1 \\ 0 \\ 0 \\ 0
    \end{bmatrix}
    u(k),
    \quad y(k) =
    \begin{bmatrix}
      1 & 0 & 0 & 0
    \end{bmatrix}
    x(k).
  \]
  \begin{itemize}
    \item[a.] Encontre a matriz de transição de estados $A^k$.
    \item[b.] Encontre $y(k)$ se $x(0) = [1 \ 1 \ 1 \ 1]^T$ e $u(k) = 0$.
    \item[c.] Encontre $y(k)$ se $x(0) = [1 \ 1 \ 1 \ 1]^T$ e $u(k) = 1$ para $k \geq 0$.
  \end{itemize}

\end{question}

% ========================= Resolução 15 =========================
\begin{resolution}

\end{resolution}

% =========================== Questão 16 ===========================
\begin{question}
  O seguinte sistema é controlável? Ele é observável?
  \[
    \begin{bmatrix}
      x_1(k+1) \\
      x_2(k+1) \\
      x_3(k+1)
    \end{bmatrix}
    =
    \begin{bmatrix}
      0  & 1  & 0  \\
      1  & 0  & 1  \\
      -1 & -2 & -3
    \end{bmatrix}
    \begin{bmatrix}
      x_1(k) \\
      x_2(k) \\
      x_3(k)
    \end{bmatrix}
    +
    \begin{bmatrix}
      1  & 0  \\
      0  & 1  \\
      -1 & -2
    \end{bmatrix}
    \begin{bmatrix}
      u_1(k) \\
      u_2(k)
    \end{bmatrix}.
  \]
  \[
    \begin{bmatrix}
      y_1(k) \\
      y_2(k)
    \end{bmatrix}
    =
    \begin{bmatrix}
      1 & -1 & 2 \\
      0 & 1  & 1
    \end{bmatrix}
    \begin{bmatrix}
      x_1(k) \\
      x_2(k) \\
      x_3(k)
    \end{bmatrix}.
  \]
\end{question}

% ========================= Resolução 16 =========================
\begin{resolution}

\end{resolution}

