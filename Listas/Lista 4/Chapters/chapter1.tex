\chapter{Resolução da Lista de Exercícios}

% =========================== Questão 01 ===========================

\begin{question}
  Encontre os autovalores e os autovetores das seguintes matrizes:
  \begin{multicols}{3}
    \begin{enumerate}[label=\alph*)]
      \item $\begin{bmatrix}
                1 & -2 \\ -2 & 1
              \end{bmatrix}$
      \item $\begin{bmatrix}
                1 & -\frac{2}{3} \\ \frac{1}{2} & \frac{1}{6}
              \end{bmatrix}$
      \item $\begin{bmatrix}
                3 & 1 \\ -1 & 1
              \end{bmatrix}$
      \item $\begin{bmatrix}
                1 & 2 \\ -1 & 1
              \end{bmatrix}$
      \item $\begin{bmatrix}
                3 & -1 & 0 \\ -1 & 2 & -1 \\ 0 & -1 & 3
              \end{bmatrix}$
      \item $\begin{bmatrix}
                -1 & -1 & 4 \\ 1 & 3 & -2 \\ 1 & 1 & -1
              \end{bmatrix}$
      \item $\begin{bmatrix}
                1 & -3 & 11 \\ 2 & -6 & 16 \\ 1 & -3 & 7
              \end{bmatrix}$
      \item $\begin{bmatrix}
                2 & -1 & -1 \\ -2 & 1 & 1 \\ 1 & 0 & 1
              \end{bmatrix}$
      \item $\begin{bmatrix}
                -4 & 4 & 2 \\ 3 & 4 & -1 \\ -3 & -2 & 3
              \end{bmatrix}$
      \item $\begin{bmatrix}
                3 & 4 & 0 & 0 \\ 4 & 3 & 0 & 0 \\ 0 & 0 & 1 & 3 \\ 0 & 0 & 4 & 5
              \end{bmatrix}$
      \item $\begin{bmatrix}
                4 & 0 & 0 & 0 \\ 1 & 3 & 0 & 0 \\ -1 & 1 & 2 & 0 \\ 1 & -1  & 1 & 1
              \end{bmatrix}$
    \end{enumerate}
  \end{multicols}
\end{question}

% ========================= Resolução 01 =========================
\begin{resolution}

\end{resolution}

% =========================== Questão 02 ===========================
\begin{question}
  \vspace{-24pt}
  \begin{enumerate}[label=\alph*)]
    \item Calcule os autovalores e os autovetores correspondentes de $A = \begin{bmatrix}
              1 & 4 & 4 \\ 3 & -1 & 0 \\ 0 & 2 & 3
            \end{bmatrix}$

    \item Calcule o traço de $A$ e verifique que é igual a soma dos autovalores.
    \item Encontre o determinante de $A$ e verifique que é igual ao produto dos autovalores.
  \end{enumerate}
\end{question}

% ========================= Resolução 02 =========================
\begin{resolution}

\end{resolution}

% =========================== Questão 03 ===========================
\begin{question}
  Encontre os autovalores e a base de cada autoespaço das seguinte matrizes:
  \begin{multicols}{3}
    \begin{enumerate}[label=\alph*)]
      \item $\begin{bmatrix}
                4 & -4 \\ 1 & 0
              \end{bmatrix}$
      \item $\begin{bmatrix}
                6 & -8 \\ 4  & -6
              \end{bmatrix}$
      \item $\begin{bmatrix}
                3 & -2 \\ 4 & -1
              \end{bmatrix}$
      \item $\begin{bmatrix}
                i & -1 \\1 & i
              \end{bmatrix}$
      \item $\begin{bmatrix}
                4 & -1 & -1 \\ 0 & 3 & 0 \\ 1 & -1 & 2
              \end{bmatrix}$
      \item $\begin{bmatrix}
                -6 & 0 & -8 \\ -4 & 2 & -4 \\ 4 & 0 & 6
              \end{bmatrix}$
      \item $\begin{bmatrix}
                -2 & 1 & -1 \\ 5 & -3 & 6 \\ 5 & -1 & 4
              \end{bmatrix}$
      \item $\begin{bmatrix}
                1 & 0 & 0 & 0 \\ 0 & 1 & 0 & 0 \\ -1 & 1 & -1 & 0 \\ 1 & 0 & -1 & 0
              \end{bmatrix}$
      \item $\begin{bmatrix}
                -1 & 0 & 1 & 2 \\ 0 & 1 & 0 & 1 \\ -1 & -4 & 1 & -2 \\ 0 & 1 & 0 & 1
              \end{bmatrix}$
    \end{enumerate}
  \end{multicols}
\end{question}

% ========================= Resolução 03 =========================
\begin{resolution}

\end{resolution}

% =========================== Questão 04 ===========================
\begin{question}
  Diagonalize as seguintes matrizes:
  \begin{multicols}{3}
    \begin{enumerate}[label=\alph*)]
      \item $\begin{bmatrix}
                3 & -9 \\ 2 & -6
              \end{bmatrix}$
      \item $\begin{bmatrix}
                5 & -4 \\ 2 & -1
              \end{bmatrix}$
      \item $\begin{bmatrix}
                -4 & -2 \\ 5 & 2
              \end{bmatrix}$
      \item $\begin{bmatrix}
                -2 & 3 & 1 \\ 0 & 1 & -1 \\ 0 & 0 & 3
              \end{bmatrix}$
      \item $\begin{bmatrix}
                8 & 0 & -3 \\ -3 & 0 & -1 \\ 3 & 0 & -2
              \end{bmatrix}$
      \item $\begin{bmatrix}
                3 & 3 & 5 \\ 5 & 6 & 5 \\ -5& -8 & -7
              \end{bmatrix}$
      \item $\begin{bmatrix}
                2 & 5 & 5 \\ 0 & 2 & 0 \\ 0 & -5 & -3
              \end{bmatrix}$
      \item $\begin{bmatrix}
                1 & 0 & -1 & 1 \\ 0 & 2 & -1 & 1 \\ 0 & 0 & -1 & 0 \\ 0 & 0 & 0 & -2
              \end{bmatrix}$
      \item $\begin{bmatrix}
                0 & 0 & 1 & 0 \\ 0 & 0 & 0 & 1 \\ 1 & 0 & 0 & 0 \\ 0 & 1 & 0 & 0
              \end{bmatrix}$
      \item $\begin{bmatrix}
                2 & 1 & -1 & 0 \\ -3 & -2 & 0 & 1 \\ 0 & 0 & 1 & -2 \\ 0 & 0 & 1 & -1
              \end{bmatrix}$
    \end{enumerate}
  \end{multicols}
\end{question}

% ========================= Resolução 04 =========================
\begin{resolution}

\end{resolution}

% =========================== Questão 05 ===========================
\begin{question}
  Escreva abaixo um matriz real que tenha:
  \begin{enumerate}[label=\alph*)]
    \item autovalores $-1$, $3$ e autovalores correspondentes $\begin{bmatrix}-1 \\ 2\end{bmatrix}, \begin{bmatrix} 1 \\ 1\end{bmatrix}$;
    \item autovalores $0,2, -2$ e autovalores correspondentes $\begin{bmatrix}-1 \\ 1 \\ 0\end{bmatrix}, \begin{bmatrix} 2 \\ -1 \\ 1\end{bmatrix}, \begin{bmatrix} 0 \\ 1 \\ 3\end{bmatrix}$;
    \item um autovalor de  $3$ e um autovalor correspondente $\begin{bmatrix}2\\ -3 \end{bmatrix}, \begin{bmatrix} 1\\ 2\end{bmatrix}$;
    \item autovalores $-1 + 2i$ e autovalor correspondente $\begin{bmatrix}1+ i \\ 3i\end{bmatrix}$;
    \item autovalores $-2$ e autovalor correspondente $\begin{bmatrix}-2 \\ 0 \\ -1\end{bmatrix}$;
  \end{enumerate}
  \vspace{8pt}
\end{question}

% ========================= Resolução 05 =========================
\begin{resolution}

\end{resolution}

% =========================== Questão 06 ===========================
\begin{question}
  Encontre uma base para o complemento ortogonal de cada um dos conjuntos a seguir no espaço de produto interno indicado.
  \begin{itemize}
    \item[(a)] $\{(0,0,0)\} \subset \mathbb{R}^3$;
    \item[(b)] $\{(1,1,1), (2,1,0)\} \subset \mathbb{R}^3$.
  \end{itemize}
\end{question}

% ========================= Resolução 06 =========================
\begin{resolution}

\end{resolution}

% =========================== Questão 07 ===========================
\begin{question}
  Calcule uma base de cada um dos quatro subespaços fundamentais das seguintes matrizes e verifique que elas satisfazem as relações de ortogonalidade:
  \begin{multicols}{2}
    \begin{itemize}
      \item[(a)] $\begin{bmatrix}
                1 & 2 \\
                3 & 6
              \end{bmatrix}$;
      \item[(b)] $\begin{bmatrix}
                2 & 1  & 3 & 1 \\
                4 & -1 & 2 & 3
              \end{bmatrix}$.
    \end{itemize}
  \end{multicols}
  \vspace{8pt}
\end{question}

% ========================= Resolução 07 =========================
\begin{resolution}

\end{resolution}

% =========================== Questão 08 ===========================
\begin{question}
  Calcule a decomposição QR da matriz:
  \[
    A = \begin{bmatrix}
      1  & 3 & 3  \\
      2  & 2 & -2 \\
      -2 & 2 & 1
    \end{bmatrix}.
  \]
\end{question}

% ========================= Resolução 08 =========================
\begin{resolution}

\end{resolution}

% =========================== Questão 09 ===========================
\begin{question}
  Calcule a decomposição em valores singulares (SVD) da matriz $A$:
  \[
    A = \begin{bmatrix}
      1  & 2 & 3 \\
      -1 & 0 & 1 \\
      3  & 2 & 1
    \end{bmatrix}.
  \]
\end{question}

% ========================= Resolução 09 =========================
\begin{resolution}

\end{resolution}

% =========================== Questão 10 ===========================
\begin{question}
  Considere o seguinte conjunto de vetores em $\mathbb{R}^2$ (com o produto interno convencional):
  \[
    S = \left\{ \mathbf{v}_1 =
    \begin{bmatrix}
      3 \\
      1
    \end{bmatrix}, \mathbf{v}_2 =
    \begin{bmatrix}
      2 \\
      2
    \end{bmatrix} \right\}.
  \]
  Agora, realize o processo de Gram-Schmidt para obter um conjunto ortogonal de vetores.
\end{question}

% ========================= Resolução 10 =========================
\begin{resolution}

\end{resolution}

% =========================== Questão 11 ===========================
\begin{question}
  Seja
  \[
    A =
    \begin{bmatrix}
      -1 & 2 & -1 \\
      0  & 1 & 3  \\
      -1 & 2 & 5
    \end{bmatrix}.
  \]
  \begin{itemize}
    \item[a.] Encontre $P(\lambda)$, o polinômio característico de $A$.
    \item[b.] Encontre os autovalores de $A$.
    \item[c.] Mostre que $P(A) = 0$.
  \end{itemize}

\end{question}

% ========================= Resolução 11 =========================
\begin{resolution}

\end{resolution}

% =========================== Questão 12 ===========================
\begin{question}
  Seja a matriz $3 \times 3$ $A$ definida como:
  \[
    A =
    \begin{bmatrix}
      0 & 1 & 0 \\
      1 & 0 & 1 \\
      0 & 1 & 0
    \end{bmatrix}.
  \]
  Calcule $A^{2004}$ e $e^{At}$.
\end{question}

% ========================= Resolução 12 =========================
\begin{resolution}

\end{resolution}

% =========================== Questão 13 ===========================
\begin{question}
  Encontre a representação no espaço de estados para os seguintes sistemas dinâmicos na forma canônica controlável:
  \begin{itemize}
    \item[a.] $\dfrac{d^3 y(t)}{dt^3} + 3 \dfrac{d^2 y(t)}{dt^2} + 4 \dfrac{dy(t)}{dt} + 2y(t) = 7u(t)$.
  \end{itemize}
\end{question}

% ========================= Resolução 13 =========================
\begin{resolution}

\end{resolution}


% =========================== Questão 14 ===========================
\begin{question}
  Determine se as seguintes matrizes são definidas positivas / semi-definidas ou definidas negativas / semi-definidas:
  \begin{multicols}{3}
    \begin{itemize}
      \item[a.] $A = \begin{bmatrix} 3 & 2 \\ 2 & 3 \end{bmatrix}$;
      \item[b.] $B = \begin{bmatrix} 1 & 2 \\ 2 & 4 \end{bmatrix}$;
      \item[c.] $C = \begin{bmatrix} 1 & -2 \\ -2 & -6 \end{bmatrix}$.
    \end{itemize}
  \end{multicols}
  \vspace{8pt}
\end{question}

% ========================= Resolução 14 =========================
\begin{resolution}

\end{resolution}


% =========================== Questão 15 ===========================
\begin{question}
  Considere o sistema no tempo discreto:
  \[
    x(k+1) =
    \begin{bmatrix}
      0 & 1 & 0 & 0 \\
      0 & 0 & 1 & 0 \\
      0 & 0 & 0 & 1 \\
      0 & 0 & 0 & 0
    \end{bmatrix}
    x(k) +
    \begin{bmatrix}
      1 \\ 0 \\ 0 \\ 0
    \end{bmatrix}
    u(k),
    \quad y(k) =
    \begin{bmatrix}
      1 & 0 & 0 & 0
    \end{bmatrix}
    x(k).
  \]
  \begin{itemize}
    \item[a.] Encontre a matriz de transição de estados $A^k$.
    \item[b.] Encontre $y(k)$ se $x(0) = [1 \ 1 \ 1 \ 1]^T$ e $u(k) = 0$.
    \item[c.] Encontre $y(k)$ se $x(0) = [1 \ 1 \ 1 \ 1]^T$ e $u(k) = 1$ para $k \geq 0$.
  \end{itemize}

\end{question}

% ========================= Resolução 15 =========================
\begin{resolution}

\end{resolution}

% =========================== Questão 16 ===========================
\begin{question}
  O seguinte sistema é controlável? Ele é observável?
  \[
    \begin{bmatrix}
      x_1(k+1) \\
      x_2(k+1) \\
      x_3(k+1)
    \end{bmatrix}
    =
    \begin{bmatrix}
      0  & 1  & 0  \\
      1  & 0  & 1  \\
      -1 & -2 & -3
    \end{bmatrix}
    \begin{bmatrix}
      x_1(k) \\
      x_2(k) \\
      x_3(k)
    \end{bmatrix}
    +
    \begin{bmatrix}
      1  & 0  \\
      0  & 1  \\
      -1 & -2
    \end{bmatrix}
    \begin{bmatrix}
      u_1(k) \\
      u_2(k)
    \end{bmatrix}.
  \]
  \[
    \begin{bmatrix}
      y_1(k) \\
      y_2(k)
    \end{bmatrix}
    =
    \begin{bmatrix}
      1 & -1 & 2 \\
      0 & 1  & 1
    \end{bmatrix}
    \begin{bmatrix}
      x_1(k) \\
      x_2(k) \\
      x_3(k)
    \end{bmatrix}.
  \]
\end{question}

% ========================= Resolução 16 =========================
\begin{resolution}

\end{resolution}

