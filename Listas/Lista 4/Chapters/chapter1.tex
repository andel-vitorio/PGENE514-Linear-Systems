\chapter{Resolução da Lista de Exercícios}

% =========================== Questão 01 ===========================

\begin{question}
  Determine se o subespaços são ortogonais.
  \begin{enumerate}[label=\alph*)]
    \item $S_1 = \operatorname{span}\left\{\begin{bmatrix}
              3 \\ 2 \\ -2
            \end{bmatrix}, \begin{bmatrix}
              0 \\ 1 \\ 0
            \end{bmatrix}\right\}, \;
            S_2 = \operatorname{span}\left\{
            \begin{bmatrix}
              2 \\ -3 \\ 0
            \end{bmatrix}
            \right\}$

    \item $S_1 = \operatorname{span}\left\{\begin{bmatrix}
              -3 \\ 0 \\ 1
            \end{bmatrix}\right\}, \;
            S_2 = \operatorname{span}\left\{
            \begin{bmatrix}
              2 \\ 1 \\ 6
            \end{bmatrix}, \;
            \begin{bmatrix}
              0 \\ 1 \\ 0
            \end{bmatrix}
            \right\}$

    \item $S_1 = \operatorname{span}\left\{\begin{bmatrix}
              1 \\ 1 \\ 1 \\1
            \end{bmatrix}\right\}, \;
            S_2 = \operatorname{span}\left\{
            \begin{bmatrix}
              -1 \\ 1 \\ -1 \\ 1
            \end{bmatrix}, \;
            \begin{bmatrix}
              0 \\ 2 \\ -2 \\ 0
            \end{bmatrix}
            \right\}$
  \end{enumerate}
\end{question}

% ========================= Resolução 01 =========================
\begin{resolution}
  Dois subespaços \( S_1 \) e \( S_2 \) são ortogonais se qualquer vetor de \( S_1 \) é ortogonal a qualquer vetor de \( S_2 \), ou seja, se o produto interno entre todos os pares de vetores de \( S_1 \) e \( S_2 \) é zero.

  \begin{enumerate}[label=\alph*)]
    \item Produto entre \(\begin{bmatrix} 3 & 2 & -2 \end{bmatrix}^{\top}\) e \(\begin{bmatrix} 2 & -3 & 0 \end{bmatrix}^{\top}\):
          \[
            \begin{bmatrix} 3 & 2 & -2 \end{bmatrix}^{\top} \cdot \begin{bmatrix} 2 & -3 & 0 \end{bmatrix}^{\top} = 3 \cdot 2 + 2 \cdot (-3) + (-2) \cdot 0 = 6 - 6 + 0 = 0.
          \]
          Produto entre \(\begin{bmatrix} 0 & 1 & 0 \end{bmatrix}^{\top}\) e \(\begin{bmatrix} 2 & -3 & 0 \end{bmatrix}^{\top}\):
          \[
            \begin{bmatrix} 0 & 1 & 0 \end{bmatrix}^{\top} \cdot \begin{bmatrix} 2 & -3 & 0 \end{bmatrix}^{\top} = 0 \cdot 2 + 1 \cdot (-3) + 0 \cdot 0 = -3.
          \]
          O produto interno do segundo par não é zero, logo, \( S_1 \) e \( S_2 \) não são ortogonais.
    \item Produto entre \(\begin{bmatrix} -3 & 0 & 1 \end{bmatrix}^{\top}\) e \(\begin{bmatrix} 2 & 1 & 6 \end{bmatrix}^{\top}\):
          \[
            \begin{bmatrix} -3 & 0 & 1 \end{bmatrix}^{\top} \cdot \begin{bmatrix} 2 & 1 & 6 \end{bmatrix}^{\top} = (-3) \cdot 2 + 0 \cdot 1 + 1 \cdot 6 = -6 + 6 = 0.
          \]
          Produto entre \(\begin{bmatrix} -3 & 0 & 1 \end{bmatrix}^{\top}\) e \(\begin{bmatrix} 0 & 1 & 0 \end{bmatrix}^{\top}\):
          \[
            \begin{bmatrix} -3 & 0 & 1 \end{bmatrix}^{\top} \cdot \begin{bmatrix} 0 & 1 & 0 \end{bmatrix}^{\top} = (-3) \cdot 0 + 0 \cdot 1 + 1 \cdot 0 = 0.
          \]
          Todos os produtos internos são zero, logo, \( S_1 \) e \( S_2 \) são ortogonais.

    \item Produto entre \(\begin{bmatrix} 1 & 1 & 1 & 1 \end{bmatrix}^{\top}\) e \(\begin{bmatrix} -1 & 1 & -1 & 1 \end{bmatrix}^{\top}\):
          \[
            \begin{bmatrix} 1 & 1 & 1 & 1 \end{bmatrix}^{\top} \cdot \begin{bmatrix} -1 & 1 & -1 & 1 \end{bmatrix}^{\top} = 1 \cdot (-1) + 1 \cdot 1 + 1 \cdot (-1) + 1 \cdot 1 = -1 + 1 - 1 + 1 = 0.
          \]
          Produto entre \(\begin{bmatrix} 1 & 1 & 1 & 1 \end{bmatrix}^{\top}\) e \(\begin{bmatrix} 0 & 2 & -2 & 0 \end{bmatrix}^{\top}\):
          \[
            \begin{bmatrix} 1 & 1 & 1 & 1 \end{bmatrix}^{\top} \cdot \begin{bmatrix} 0 & 2 & -2 & 0 \end{bmatrix}^{\top} = 1 \cdot 0 + 1 \cdot 2 + 1 \cdot (-2) + 1 \cdot 0 = 0.
          \]
          Todos os produtos internos são zero, logo, \( S_1 \) e \( S_2 \) são ortogonaisx.
  \end{enumerate}
\end{resolution}

% =========================== Questão 02 ===========================
\begin{question}
  Encontre a projeção do vetor $v$ sobre o espaço $S$:
  \begin{enumerate}[label=\alph*)]
    \item $S = \operatorname{span}\left\{
            \begin{bmatrix}
              0 \\ 0 \\ -1 \\ 1
            \end{bmatrix}, \;
            \begin{bmatrix}
              0 \\ 1 \\ 1 \\ 1
            \end{bmatrix}
            \right\}$, $v = \begin{bmatrix}
              1 \\ 0 \\ 1 \\ 1
            \end{bmatrix}$.

    \item $S = \operatorname{span}\left\{
            \begin{bmatrix}
              -1 \\ 2 \\ 0 \\ 0
            \end{bmatrix}, \;
            \begin{bmatrix}
              0 \\ 0 \\ 1 \\ 0
            \end{bmatrix}, \;
            \begin{bmatrix}
              0 \\ 0 \\ 0 \\ 1
            \end{bmatrix}
            \right\}$, $v = \begin{bmatrix}
              1 \\ 1 \\ 1 \\ 1
            \end{bmatrix}$.
  \end{enumerate}
\end{question}

% ========================= Resolução 02 =========================
\begin{resolution}
  A projeção de um vetor \( v \) sobre um subespaço \( S \) pode ser obtida resolvendo o sistema linear associado à projeção ortogonal. Se os vetores \(\{u_1, u_2, \ldots, u_k\}\) são geradores de \( S \), a projeção de \( v \) sobre $S$ é dada por:

  \begin{equation}
    \operatorname{proj}_S(v) = \sum\limits_{i=1}^k \operatorname{proj}_{u_i}(v) = U (U^T U)^{-1} U^T v,
  \end{equation}
  onde \( U \) é a matriz cujas colunas são os vetores geradores de \( S \).

  \begin{enumerate}[label=\alph*)]
    \item Dada a matriz \( U \):
          \[
            U = \begin{bmatrix}
              0  & 0 \\
              0  & 1 \\
              -1 & 1 \\
              1  & 1
            \end{bmatrix},
          \]
          o produto \( U^T U \) é:
          \[
            U^T U = \begin{bmatrix}
              0 & 0 & -1 & 1 \\
              0 & 1 & 1  & 1
            \end{bmatrix}
            \begin{bmatrix}
              0  & 0 \\
              0  & 1 \\
              -1 & 1 \\
              1  & 1
            \end{bmatrix}
            = \begin{bmatrix}
              2 & 0 \\
              0 & 3
            \end{bmatrix}.
          \]
          A inversa de \( U^T U \) é dada por:
          \[
            (U^T U)^{-1} = \begin{bmatrix}
              \frac{1}{2} & 0           \\
              0           & \frac{1}{3}
            \end{bmatrix}.
          \]
          O produto \( U^T v \), onde \( v = \begin{bmatrix} 1 & 0 & 1 & 1 \end{bmatrix}^{\top} \), é:
          \[
            U^T v = \begin{bmatrix}
              0 & 0 & -1 & 1 \\
              0 & 1 & 1  & 1
            \end{bmatrix}
            \begin{bmatrix}
              1 \\ 0 \\ 1 \\ 1
            \end{bmatrix}
            = \begin{bmatrix}
              0 \\ 2
            \end{bmatrix}.
          \]
          Assim, a projeção de \( v \) no subespaço gerado pelas colunas de \( U \), denotada por \( \operatorname{proj}_S(v) \), é calculada como:
          \[
            \operatorname{proj}_S(v) = U (U^T U)^{-1} U^T v = U \begin{bmatrix}
              \frac{1}{2} & 0           \\
              0           & \frac{1}{3}
            \end{bmatrix} \begin{bmatrix}
              0 \\ 2
            \end{bmatrix}.
          \]
          Primeiro, calculamos:
          \[
            (U^T U)^{-1} U^T v = \begin{bmatrix}
              \frac{1}{2} & 0           \\
              0           & \frac{1}{3}
            \end{bmatrix}
            \begin{bmatrix}
              0 \\ 2
            \end{bmatrix}
            = \begin{bmatrix}
              0 \\ \frac{2}{3}
            \end{bmatrix}.
          \]
          Em seguida, multiplicamos pela matriz \( U \):
          \[
            \operatorname{proj}_S(v) = U \begin{bmatrix}
              0 \\ \frac{2}{3}
            \end{bmatrix}
            = \begin{bmatrix}
              0 \\ \frac{2}{3} \\[8pt] \frac{2}{3} \\[8pt] \frac{2}{3}
            \end{bmatrix}.
          \]

    \item Considerando a matriz \( U \):
          \[
            U = \begin{bmatrix}
              -1 & 0 & 0 \\
              2  & 0 & 0 \\
              0  & 1 & 0 \\
              0  & 0 & 1
            \end{bmatrix},
          \]
          o produto \( U^T U \) resulta em:
          \[
            U^T U = \begin{bmatrix}
              5 & 0 & 0 \\
              0 & 1 & 0 \\
              0 & 0 & 1
            \end{bmatrix}.
          \]
          A inversa de \( U^T U \) é:
          \[
            (U^T U)^{-1} = \begin{bmatrix}
              \frac{1}{5} & 0 & 0 \\
              0           & 1 & 0 \\
              0           & 0 & 1
            \end{bmatrix}.
          \]
          O produto \( U^T v \), com \( v = \begin{bmatrix} 1 & 1 & 1 & 1 \end{bmatrix}^{\top} \), é:
          \[
            U^T v = \begin{bmatrix}
              -1 & 2 & 0 & 0 \\
              0  & 0 & 1 & 0 \\
              0  & 0 & 0 & 1
            \end{bmatrix}
            \begin{bmatrix}
              1 \\ 1 \\ 1 \\ 1
            \end{bmatrix}
            = \begin{bmatrix}
              1 \\ 1 \\ 1
            \end{bmatrix}.
          \]
          A projeção \( \operatorname{proj}_S(v) \) é dada por:
          \[
            \operatorname{proj}_S(v) = U (U^T U)^{-1} U^T v = U \begin{bmatrix}
              \frac{1}{5} & 0 & 0 \\
              0           & 1 & 0 \\
              0           & 0 & 1
            \end{bmatrix} \begin{bmatrix}
              1 \\ 1 \\ 1
            \end{bmatrix}.
          \]
          Calculando:
          \[
            (U^T U)^{-1} U^T v = \begin{bmatrix}
              \frac{1}{5} \\ 1 \\ 1
            \end{bmatrix}.
          \]
          Finalmente, temos:
          \[
            \operatorname{proj}_S(v) = U \begin{bmatrix}
              \frac{1}{5} \\ 1 \\ 1
            \end{bmatrix}
            = \begin{bmatrix}
              -\frac{1}{5} \\ \frac{2}{5} \\ 1 \\ 1
            \end{bmatrix}.
          \]
  \end{enumerate}

\end{resolution}

% =========================== Questão 03 ===========================
\begin{question}
  Encontre as bases para os quatro subespaços fundamentais  da matriz $A$:
  \begin{multicols}{2}
    \begin{enumerate}[label=\alph*)]
      \item $A = \begin{bmatrix}
                1 & 2 & 3 \\ 0 & 1 & 0
              \end{bmatrix}$
      \item $A = \begin{bmatrix}
                1 & 0 & 0 \\ 0 & 1 & 1 \\ 1 & 1 & 1 \\ 1 & 2 & 2
              \end{bmatrix}$
      \item $A = \begin{bmatrix}
                0 & -1 & 1 \\ 1 & 2 & 0 \\ 1 & 1 & 1
              \end{bmatrix}$
      \item $A = \begin{bmatrix}
                1 & 0 & -1 \\ 0 & -1 & 1 \\ 1 & 1 & 0 \\ 1 & 0 & 1
              \end{bmatrix}$
    \end{enumerate}
  \end{multicols}
\end{question}

% ========================= Resolução 03 =========================
\begin{resolution}
  Para determinar as bases dos quatro subespaços fundamentais de uma matriz \( A \) (\( C(A) \), \( N(A) \), \( C(A^T) \), \( N(A^T) \)), os seguintes passos são realizados:

  \begin{itemize}
    \item Espaço Nulo \( \operatorname{ker}(A) \): Resolve-se a equação \( A x = 0 \) para encontrar os vetores que compõem o espaço nulo.
    \item Espaço Coluna \( C(A) \): Identificam-se as colunas linearmente independentes de \( A \), utilizando escalonamento ou decomposição.
    \item Espaço Linha \( C(A^T) \): Identificam-se as linhas linearmente independentes de \( A \), que correspondem às colunas linearmente independentes de \( A^T \).
    \item Espaço Nulo à Esquerda \( \operatorname{ker}(A^T) \): Resolve-se a equação \( A^T y = 0 \).
  \end{itemize}

  \begin{enumerate}[label=\alph*)]
    \item Para determinar o espaço coluna (\( C(A) \)), encontra-se as colunas independentes por meio da eliminação:
          \[
            C(A) = \operatorname{span}\left\{
            \begin{bmatrix} 1 \\ 0 \end{bmatrix},
            \begin{bmatrix} 2 \\ 1 \end{bmatrix}
            \right\}.
          \]
          Para determinar o espaço nulo (\( \operatorname{ker}(A) \)), resolve-se \( A x = 0 \):
          \[
            \begin{bmatrix}
              1 & 2 & 3 \\
              0 & 1 & 0
            \end{bmatrix}
            \begin{bmatrix}
              x_1 \\ x_2 \\ x_3
            \end{bmatrix} =
            \begin{bmatrix}
              0 \\ 0
            \end{bmatrix}.
          \]
          O que leva às equações:
          \[
            x_1 + 2x_2 + 3x_3 = 0, \quad x_2 = 0.
          \]
          Logo:
          \[
            x_1 = -3x_3, \quad x_2 = 0, \quad x_3 = x_3.
          \]
          A base é:
          \[
            \operatorname{ker}(A) = \operatorname{span}\left\{
            \begin{bmatrix}
              -3 \\ 0 \\ 1
            \end{bmatrix}
            \right\}.
          \]
          Para determinar o espaço linha (\( C(A^T) \)), segue-se o mesmo procedimento:
          \[
            C(A^T) = \operatorname{span}\left\{
            \begin{bmatrix} 1 \\ 2 \\ 3 \end{bmatrix},
            \begin{bmatrix} 0 \\ 1 \\ 0 \end{bmatrix}
            \right\}.
          \]
          Para determinar o espaço nulo à esquerda (\( \operatorname{ker}(A^T) \)), resolve-se \( A^T y = 0 \):
          \[
            \begin{bmatrix}
              1 & 0 \\
              2 & 1 \\
              3 & 0
            \end{bmatrix}
            \begin{bmatrix}
              y_1 \\ y_2
            \end{bmatrix} =
            \begin{bmatrix}
              0 \\ 0 \\ 0
            \end{bmatrix}.
          \]
          O que implica que:
          \[
            y_1 = 0, \quad y_2 = 0.
          \]
          Logo:
          \[
            \operatorname{ker} (A^T) = \{0\}.
          \]

    \item Para determinar o espaço coluna (\( C(A) \)), encontra-se as colunas independentes por meio da eliminação:
          \[
            C(A) = \operatorname{span}\left\{
            \begin{bmatrix} 0 \\ 1 \\ 1 \end{bmatrix},
            \begin{bmatrix} -1 \\ 2 \\ 1 \end{bmatrix}
            \right\}.
          \]
          Para determinar o espaço nulo (\( \operatorname{ker}(A) \)), resolve-se \( A x = 0 \):
          \[
            \begin{bmatrix}
              0 & -1 & 1 \\
              1 & 2  & 0 \\
              1 & 1  & 1
            \end{bmatrix}
            \begin{bmatrix}
              x_1 \\ x_2 \\ x_3
            \end{bmatrix} =
            \begin{bmatrix}
              0 \\ 0 \\ 0
            \end{bmatrix}.
          \]
          O que leva às equações:
          \[
            -x_2 + x_3 = 0, \quad x_1 + x_2 = 0, \quad x_1 + x_2 + x_3 = 0.
          \]
          Logo:
          \[
            x_1 = 0, \quad x_2 = 0, \quad x_3 = 0.
          \]
          A base é:
          \[
            \operatorname{ker}(A) = \{0\}.
          \]
          Para determinar o espaço linha (\( C(A^T) \)), segue-se o mesmo procedimento:
          \[
            C(A^T) = \operatorname{span}\left\{
            \begin{bmatrix} 0 \\ -1 \\ 1 \end{bmatrix},
            \begin{bmatrix} 1 \\ 2 \\ 0 \end{bmatrix}
            \right\}.
          \]
          Para determinar o espaço nulo à esquerda (\( \operatorname{ker}(A^T) \)), resolve-se \( A^T y = 0 \):
          \[
            \begin{bmatrix}
              0  & 1 & 1 \\
              -1 & 2 & 1 \\
              1  & 0 & 1
            \end{bmatrix}
            \begin{bmatrix}
              y_1 \\ y_2 \\ y_3
            \end{bmatrix} =
            \begin{bmatrix}
              0 \\ 0 \\ 0
            \end{bmatrix}.
          \]
          O que leva às equações:
          \[
            y_2 + y_3 = 0, \quad - y_1 - 2 y_2 + y_3 = 0, \quad y_1 + y_3 = 0.
          \]
          Logo:
          \[
            \operatorname{ker}(A^T) = \{0\}.
          \]

    \item Para determinar o espaço coluna (\( C(A) \)), encontra-se as colunas independentes por meio da eliminação:
          \[
            C(A) = \operatorname{span}\left\{
            \begin{bmatrix} 1 \\ 0 \\ 1 \\ 1 \end{bmatrix},
            \begin{bmatrix} 0 \\ 1 \\ 1 \\ 2 \end{bmatrix}
            \right\}.
          \]
          Para determinar o espaço nulo (\( \operatorname{ker}(A) \)), resolve-se \( A x = 0 \):
          \[
            \begin{bmatrix}
              1 & 0 & 0 \\
              0 & 1 & 1 \\
              1 & 1 & 1 \\
              1 & 2 & 2
            \end{bmatrix}
            \begin{bmatrix}
              x_1 \\ x_2 \\ x_3
            \end{bmatrix} =
            \begin{bmatrix}
              0 \\ 0 \\ 0 \\ 0
            \end{bmatrix}.
          \]
          O que leva às equações:
          \[
            x_1 = 0, \quad x_2 + x_3 = 0, \quad x_1 + x_2 + x_3 = 0, \quad x_1 + 2x_2 + 2x_3 = 0.
          \]
          Logo:
          \[
            x_1 = 0, \quad x_2 = -x_3, \quad x_3 = x_3.
          \]
          A base é:
          \[
            \operatorname{ker}(A) =  \operatorname{span}\left\{
            \begin{bmatrix} 0 \\ -1 \\ 1 \end{bmatrix}
            \right\}.
          \]
          Para determinar o espaço linha (\( C(A^T) \)), segue-se o mesmo procedimento:
          \[
            C(A^T) = \operatorname{span}\left\{
            \begin{bmatrix} 1 \\ 0 \\ 0 \end{bmatrix},
            \begin{bmatrix} 0 \\ 1 \\ 1 \end{bmatrix}
            \right\}.
          \]
          Para determinar o espaço nulo à esquerda (\( \operatorname{ker}(A^T) \)), resolve-se \( A^T y = 0 \):
          \[
            \begin{bmatrix}
              1

                & 0 & 1 & 1 \\
              0 & 1 & 1 & 2 \\
              0 & 0 & 0 & 0
            \end{bmatrix}
            \begin{bmatrix}
              y_1 \\ y_2 \\ y_3 \\ y_4
            \end{bmatrix} =
            \begin{bmatrix}
              0 \\ 0 \\ 0 \\ 0
            \end{bmatrix}.
          \]
          O que leva à equação:
          \[
            y_1 = -y_3 - y_4, \quad y_2 = -y_3 - 2y_4.
          \]
          Logo:
          \[
            \operatorname{ker}(A^T) = \operatorname{span}\left\{
            \begin{bmatrix} -1 \\ -1 \\ 1 \\ 0 \end{bmatrix},
            \begin{bmatrix} -1 \\ -2 \\ 0 \\ 1 \end{bmatrix}
            \right\}.
          \]
  \end{enumerate}
\end{resolution}

% =========================== Questão 04 ===========================
\begin{question}
  Encontre as solução dos mínimos quadrados do sistemas $Ax = b$.
  \begin{multicols}{2}
    \begin{enumerate}[label=\alph*)]
      \item $A = \begin{bmatrix}
                2 & 1 \\ 1 & 2 \\ 1 & 1
              \end{bmatrix}$, $b = \begin{bmatrix}
                2 \\ 0 \\ -3
              \end{bmatrix}$.
      \item $A = \begin{bmatrix}
                1 & -1 & 1 \\ 1 & 1 & 1 \\ 0 & 1 & 1 \\ 1 & 0 & 1
              \end{bmatrix}$, $b = \begin{bmatrix}
                2 \\ 1 \\ 0 \\ 2
              \end{bmatrix}$.
    \end{enumerate}
  \end{multicols}
\end{question}

% ========================= Resolução 04 =========================
\begin{resolution}
  A solução dos mínimos quadrados de um sistema \( Ax = b \) é dada pela minimização da norma \( \|Ax - b\|^2 \), o que resulta na equação normal:
  \[
    A^T A x = A^T b.
  \]

  \begin{enumerate}[label=\alph*)]
    \item O primeiro passo consiste em calcular \( A^T A \):
          \[
            A^T A = \begin{bmatrix}
              2 & 1 & 1 \\
              1 & 2 & 1
            \end{bmatrix}
            \begin{bmatrix}
              2 & 1 \\
              1 & 2 \\
              1 & 1
            \end{bmatrix} =
            \begin{bmatrix}
              6 & 5 \\
              5 & 6
            \end{bmatrix}.
          \]

          Em seguida, calcula-se \( A^T b \):
          \[
            A^T b = \begin{bmatrix}
              2 & 1 & 1 \\
              1 & 2 & 1
            \end{bmatrix}
            \begin{bmatrix}
              2 \\
              0 \\
              -3
            \end{bmatrix} =
            \begin{bmatrix}
              1 \\
              -1
            \end{bmatrix}.
          \]

          Por fim, resolve-se o sistema \( A^T A x = A^T b \):
          \[
            \begin{bmatrix}
              6 & 5 \\
              5 & 6
            \end{bmatrix}
            \begin{bmatrix}
              x_1 \\
              x_2
            \end{bmatrix} =
            \begin{bmatrix}
              1 \\
              -1
            \end{bmatrix}.
          \]
          O sistema resultante é:
          \[
            6x_1 + 5x_2 = 1, \quad 5x_1 + 6x_2 = -1.
          \]
          A solução obtida por substituição é:
          \[
            x_1 = 1, \quad x_2 = -1.
          \]

          Portanto, a solução é:
          \[
            x = \begin{bmatrix}
              1 \\
              -1
            \end{bmatrix}.
          \]

    \item Inicialmente, calcula-se \( A^T A \):
          \[
            A^T A = \begin{bmatrix}
              1  & 1 & 0 & 1 \\
              -1 & 1 & 1 & 0 \\
              1  & 1 & 1 & 1
            \end{bmatrix}
            \begin{bmatrix}
              1 & -1 & 1 \\
              1 & 1  & 1 \\
              0 & 1  & 1 \\
              1 & 0  & 1
            \end{bmatrix} =
            \begin{bmatrix}
              3 & 0 & 3 \\
              0 & 3 & 1 \\
              3 & 1 & 4
            \end{bmatrix}.
          \]

          Em seguida, calcula-se \( A^T b \):
          \[
            A^T b = \begin{bmatrix}
              1  & 1 & 0 & 1 \\
              -1 & 1 & 1 & 0 \\
              1  & 1 & 1 & 1
            \end{bmatrix}
            \begin{bmatrix}
              2 \\
              1 \\
              0 \\
              2
            \end{bmatrix} =
            \begin{bmatrix}
              5  \\
              -1 \\
              5
            \end{bmatrix}.
          \]

          Finalmente, resolve-se o sistema \( A^T A x = A^T b \):
          \[
            \begin{bmatrix}
              3 & 0 & 3 \\
              0 & 3 & 1 \\
              3 & 1 & 4
            \end{bmatrix}
            \begin{bmatrix}
              x_1 \\
              x_2 \\
              x_3
            \end{bmatrix} =
            \begin{bmatrix}
              5  \\
              -1 \\
              5
            \end{bmatrix}.
          \]
          O sistema resultante é:
          \[
            \begin{aligned}
              3x_1 + 3x_3       & = 5,  \\
              3x_2 + x_3        & = -1, \\
              3x_1 + x_2 + 4x_3 & = 5.
            \end{aligned}
          \]
          A solução obtida por substituição é:
          \[
            x_1 = \frac{7}{6}, \quad x_2 = -\frac{1}{2}, \quad x_3 = \frac{1}{2}.
          \]

          Portanto, a solução é:
          \[
            x = \begin{bmatrix}
              \frac{7}{6}  \\[8pt]
              -\frac{1}{2} \\[8pt]
              \frac{1}{2}
            \end{bmatrix}.
          \]

  \end{enumerate}
\end{resolution}

% =========================== Questão 05 ===========================
\begin{question}
  Encontre a complemente ortogonal $S^{\perp}$ e encontre a soma direta $S \oplus S^{\perp}$:
  \begin{multicols}{2}
    \begin{enumerate}[label=\alph*)]
      \item $S = \operatorname{span}\left\{
              \begin{bmatrix} 0 \\ 1 \\ 0 \end{bmatrix},
              \begin{bmatrix} 2 \\ 0 \\ 1 \end{bmatrix}
              \right\}$.
      \item $S = \operatorname{span}\left\{
              \begin{bmatrix} 0 \\ 1 \\ -1 \\ 1 \end{bmatrix}
              \right\}$.
      \item $= \operatorname{span}\left\{
              \begin{bmatrix} 0 \\ -2 \\ 1 \end{bmatrix}
              \right\}$.
      \item $ S = \operatorname{span}\left\{
              \begin{bmatrix} 0 \\ 1 \\ -1 \\ 1 \\ -1 \end{bmatrix},
              \begin{bmatrix} 0 \\ 1 \\ 0 \\ 2 \\ -1 \end{bmatrix},
              \begin{bmatrix} 2 \\ 0 \\ 1 \\ 0 \\ 2 \end{bmatrix}
              \right\}$.
    \end{enumerate}
  \end{multicols}
  \vspace{8pt}
\end{question}

% ========================= Resolução 05 =========================
\begin{resolution}
  Para determinar o complemento ortogonal \( S^\perp \), é necessário encontrar o conjunto de vetores \( x \) tais que \( x \cdot s = 0 \) para todos os vetores \( s \in S \). Isso equivale a resolver \( A^T x = 0 \), onde \( A \) é a matriz cujas colunas são os geradores de \( S \). A soma direta \( S \oplus S^\perp \) equivale ao espaço completo, com \( \dim(S) + \dim(S^\perp) = \dim(\text{espaço total}) \).

  \begin{enumerate}[label=\alph*)]
    \item Primeiramente, constrói-se a matriz \( A \):
          \[
            A = \begin{bmatrix}
              0 & 2 \\
              1 & 0 \\
              0 & 1
            \end{bmatrix}.
          \]

          Em seguida, resolve-se \( A^T x = 0 \):
          O sistema é:
          \[
            \begin{bmatrix}
              0 & 1 & 0 \\
              2 & 0 & 1
            \end{bmatrix}
            \begin{bmatrix}
              x_1 \\
              x_2 \\
              x_3
            \end{bmatrix}
            = 0.
          \]

          Expandindo:
          \[
            \begin{aligned}
              x_2        & = 0,                      \\
              2x_1 + x_3 & = 0 \implies x_3 = -2x_1.
            \end{aligned}
          \]

          Logo, o complemento ortogonal \( S^\perp \):
          \[
            S^\perp = \operatorname{span}\left\{
            \begin{bmatrix}
              1 \\ 0 \\ -2
            \end{bmatrix}
            \right\}.
          \]

          E a soma direta \( S \oplus S^\perp \):
          Como \( S \) tem dimensão 2 e \( S^\perp \) tem dimensão 1, temos:
          \[
            S \oplus S^\perp = \mathbb{R}^3.
          \]


    \item  Construindo a matriz \( A \):
          \[
            A = \begin{bmatrix}
              0  \\
              1  \\
              -1 \\
              1
            \end{bmatrix}.
          \]

          Resolvendo \( A^T x = 0 \):
          \[
            \begin{bmatrix}
              0 & 1 & -1 & 1
            \end{bmatrix}
            \begin{bmatrix}
              x_1 \\
              x_2 \\
              x_3 \\
              x_4
            \end{bmatrix}
            = 0.
          \]

          Expandindo:
          \[
            x_2 - x_3 + x_4 = 0 \Rightarrow x_2 = x_3 + x_4.
          \]

          Complemento ortogonal \( S^\perp \):
          \( S^\perp \) é o espaço das soluções do sistema, dado por:
          \[
            S^\perp = \operatorname{span}\left\{
            \begin{bmatrix}
              0 \\ 1 \\ 1 \\ 0
            \end{bmatrix},
            \begin{bmatrix}
              0 \\ 1 \\ 0 \\ 1
            \end{bmatrix}
            \right\}.
          \]

          Soma direta \( S \oplus S^\perp \):
          Como \( S \) tem dimensão 1 e \( S^\perp \) tem dimensão 2, temos:
          \[
            S \oplus S^\perp = \mathbb{R}^3.
          \]

    \item Construindo-se a matriz \( A \):
          \[
            A = \begin{bmatrix}
              0  \\
              -2 \\
              1
            \end{bmatrix}.
          \]

          Resolver \( A^T x = 0 \):
          O sistema é:
          \[
            \begin{bmatrix}
              0 & -2 & 1
            \end{bmatrix}
            \begin{bmatrix}
              x_1 \\
              x_2 \\
              x_3
            \end{bmatrix}
            = 0.
          \]

          Expandindo:
          \[
            -2x_2 + x_3 = 0 \implies x_3 = 2x_2.
          \]

          Complemento ortogonal \( S^\perp \):
          \[
            S^\perp = \operatorname{span}\left\{
            \begin{bmatrix}
              1 \\ 0 \\ 0
            \end{bmatrix},
            \begin{bmatrix}
              0 \\ 1 \\ 2
            \end{bmatrix}
            \right\}.
          \]

          Soma direta \( S \oplus S^\perp \):
          Como \( S \) tem dimensão 1 e \( S^\perp \) tem dimensão 2, temos:
          \[
            S \oplus S^\perp = \mathbb{R}^3.
          \]

    \item Construindo-se a matriz \( A \):
          \[
            A = \begin{bmatrix}
              0  & 0  & 2 \\
              1  & 1  & 0 \\
              -1 & 0  & 1 \\
              1  & 2  & 0 \\
              -1 & -1 & 2
            \end{bmatrix}.
          \]

          Resolver \( A^T x = 0 \):
          O sistema é:
          \[
            A^T x = 0.
          \]
          Calculando \( S^\perp \) (usando escalonamento ou outras técnicas), temos:

          Complemento ortogonal \( S^\perp \):
          \[
            S^\perp = \operatorname{span}\left\{
            \begin{bmatrix}
              1 \\ 0 \\ 0 \\ 0 \\ 0
            \end{bmatrix},
            \begin{bmatrix}
              0 \\ 1 \\ 0 \\ 0 \\ 0
            \end{bmatrix},
            \begin{bmatrix}
              0 \\ 0 \\ 1 \\ 0 \\ 0
            \end{bmatrix}
            \right\}.
          \]

          Soma direta \( S \oplus S^\perp \):
          Como \( S \) tem dimensão 3 e \( S^\perp \) tem dimensão 3, temos:
          \[
            S \oplus S^\perp = \mathbb{R}^6.
          \]
  \end{enumerate}
\end{resolution}

% =========================== Questão 06 ===========================
\begin{question}
  Encontre a fatoração QR de cada matriz:
  \begin{multicols}{3}
    \begin{enumerate}[label=\alph*)]
      \item $\begin{bmatrix}
                1 & 1 \\ 0 & 1 \\ 1 & 0
              \end{bmatrix}$.
      \item $\begin{bmatrix}
                1 & 0 \\ 0 & 0 \\ 1 & 1 \\ 1 & 2
              \end{bmatrix}$.
      \item $\begin{bmatrix}
                1 & 0 & -1 \\ 1 & 2 & 0 \\ 1 & 2 & 0 \\ 1 & 0 & 0
              \end{bmatrix}$
    \end{enumerate}
  \end{multicols}
  \vspace{8pt}
\end{question}

% ========================= Resolução 06 =========================
\begin{resolution}
  Por meio do algoritmo implementado no trabalho pedido, temos:
  \begin{enumerate}[label=\alph*)]
    \item $\begin{bmatrix}
              1 & 1 \\ 0 & 1 \\ 1 & 0
            \end{bmatrix} = \begin{bmatrix}
              0.71 & 0.41  \\
              0.   & 0.82  \\
              0.71 & -0.41 \\
            \end{bmatrix} \begin{bmatrix}
              1.41 & 0.71 \\
              0.   & 1.22
            \end{bmatrix}$.
    \item $\begin{bmatrix}
              1 & 0 \\ 0 & 0 \\ 1 & 1 \\ 1 & 2
            \end{bmatrix} = \begin{bmatrix}
              0.58 & -0.71 \\
              0.   & 0.    \\
              0.58 & 0.    \\
              0.58 & 0.71  \\
            \end{bmatrix} \begin{bmatrix}
              1.73 & 1.73 \\
              0.   & 1.41
            \end{bmatrix}$.
    \item $\begin{bmatrix}
              1 & 0 & -1 \\ 1 & 2 & 0 \\ 1 & 2 & 0 \\ 1 & 0 & 0
            \end{bmatrix} = \begin{bmatrix}
              0.5 & -0.5 & -0.71 \\
              0.5 & 0.5  & 0.    \\
              0.5 & 0.5  & 0.    \\
              0.5 & -0.5 & 0.71
            \end{bmatrix} \begin{bmatrix}
              2. & 2. & -0.5 \\
              0. & 2. & 0.5  \\
              0. & 0. & 0.71
            \end{bmatrix}$.
  \end{enumerate}
\end{resolution}

% =========================== Questão 07 ===========================
\begin{question}
  Determine se o conjunto de vetores em $\mathbb{R}^{n}$ é ortogonal. Se o conjunto é ortogonal, então determine se ele é ortonormal também e se o conjunto é base para $\mathbb{R}^{n}$.
  \begin{enumerate}[label=\alph*)]
    \item $\{(2, -4), (2, 1)\}$
    \item $\{(-2, 5), (4, 0)\}$
    \item $\left\{\left(\frac{4}{5}, \frac{4}{5}\right), \left(\frac{-4}{5}, \frac{3}{5}\right)\right\}$
    \item $\left\{(2, 1), \left(\frac{1}{3}, -\frac{2}{3}\right)\right\}$
    \item $\left\{(4, -1, 1), (-1, 0, 4), (-4, -17, -1)\right\}$
    \item $\left\{(2, -4, 2), (0, 2, 4), (-10, -4, 2)\right\}$
    \item $\left\{\left(\frac{\sqrt{2}}{3}, 0, -\frac{\sqrt{2}}{6}\right), \left(0, \frac{2\sqrt{5}}{5}, -\frac{\sqrt{5}}{5}\right), \left(\frac{\sqrt{5}}{5}, 0, \frac{1}{2}\right)\right\}$
    \item $\left\{\left(\frac{\sqrt{2}}{3}, 0, \frac{\sqrt{2}}{2}\right), \left(-\frac{\sqrt{6}}{6}, \frac{\sqrt{6}}{3}, \frac{\sqrt{6}}{6}\right), \left(\frac{\sqrt{3}}{3}, \frac{\sqrt{3}}{3}, -\frac{\sqrt{3}}{3}\right)\right\}$
    \item $\left\{(2, 5, -3), (4, 2, 6)\right\}$
    \item $\left\{(-6, 3, 2, 1), (2, 0, 6, 0)\right\}$
  \end{enumerate}
\end{question}

% ========================= Resolução 07 =========================
\begin{resolution}
  Para verificar se os conjuntos são ortogonais, calcula-se o produto escalar entre os vetores. Se o produto escalar entre todos os pares de vetores for zero, o conjunto é ortogonal. Se o conjunto for ortogonal, verifica-se se é ortonormal ao calcular o módulo (\(\|v\|\)) de cada vetor. Para ser ortonormal, todos os vetores devem ter módulo igual a 1. Se o conjunto tem \(n\) vetores linearmente independentes em \(\mathbb{R}^n\), ele forma uma base para \(\mathbb{R}^n\).

  \begin{enumerate}[label=\alph*)]
    \item O produto escalar entre os vetores \((2, -4)\) e \((2, 1)\) é calculado da seguinte forma:
          \[
            (2, -4) \cdot (2, 1) = 2 \cdot 2 + (-4) \cdot 1 = 4 - 4 = 0.
          \]
          Portanto, os vetores são ortogonais. O módulo dos vetores é dado por:
          \[
            \|(2, -4)\| = \sqrt{2^2 + (-4)^2} = \sqrt{4 + 16} = \sqrt{20},
          \]
          \[
            \|(2, 1)\| = \sqrt{2^2 + 1^2} = \sqrt{4 + 1} = \sqrt{5}.
          \]
          Como os vetores não possuem módulo unitário, o conjunto não é ortonormal. Considerando que existem dois vetores em \(\mathbb{R}^2\) e eles são linearmente independentes, o conjunto forma uma base para \(\mathbb{R}^2\).

    \item O produto escalar entre os vetores \((-2, 5)\) e \((4, 0)\) é calculado como:
          \[
            (-2, 5) \cdot (4, 0) = -2 \cdot 4 + 5 \cdot 0 = -8.
          \]
          Assim, o conjunto não é ortogonal.

    \item O produto escalar entre os vetores \(\left(\frac{4}{5}, \frac{4}{5}\right)\) e \(\left(\frac{-4}{5}, \frac{3}{5}\right)\) resulta em:
          \[
            \left(\frac{4}{5}, \frac{4}{5}\right) \cdot \left(\frac{-4}{5}, \frac{3}{5}\right) = \frac{4}{5} \cdot \frac{-4}{5} + \frac{4}{5} \cdot \frac{3}{5} = -\frac{16}{25} + \frac{12}{25} = -\frac{4}{25}.
          \]
          Portanto, o conjunto não é ortogonal.

    \item O produto escalar entre \((2, 1)\) e \(\left(\frac{1}{3}, -\frac{2}{3}\right)\) é dado por:
          \[
            (2, 1) \cdot \left(\frac{1}{3}, -\frac{2}{3}\right) = 2 \cdot \frac{1}{3} + 1 \cdot \left(-\frac{2}{3}\right) = \frac{2}{3} - \frac{2}{3} = 0.
          \]
          Assim, os vetores são ortogonais. O módulo dos vetores é calculado como:
          \[
            \|(2, 1)\| = \sqrt{2^2 + 1^2} = \sqrt{4 + 1} = \sqrt{5},
          \]
          \[
            \left\|\left(\frac{1}{3}, -\frac{2}{3}\right)\right\| = \sqrt{\left(\frac{1}{3}\right)^2 + \left(-\frac{2}{3}\right)^2} = \sqrt{\frac{1}{9} + \frac{4}{9}} = \sqrt{\frac{5}{9}} = \frac{\sqrt{5}}{3}.
          \]
          Como os vetores não têm módulo unitário, o conjunto não é ortonormal. Considerando que existem dois vetores em \(\mathbb{R}^2\) e eles são linearmente independentes, o conjunto forma uma base para \(\mathbb{R}^2\).

    \item Os produtos escalares entre os vetores \((4, -1, 1)\), \((-1, 0, 4)\) e \((-4, -17, -1)\) são os seguintes:
          \begin{itemize}
            \item \((4, -1, 1) \cdot (-1, 0, 4) = -4 + 0 + 4 = 0\),
            \item \((4, -1, 1) \cdot (-4, -17, -1) = -16 + 17 - 1 = 0\),
            \item \((-1, 0, 4) \cdot (-4, -17, -1) = 4 + 0 - 4 = 0\).
          \end{itemize}
          O conjunto é ortogonal. O módulo dos vetores é dado por:
          \begin{itemize}
            \item \(\|(4, -1, 1)\| = \sqrt{4^2 + (-1)^2 + 1^2} = \sqrt{16 + 1 + 1} = \sqrt{18}\),
            \item \(\|(-1, 0, 4)\| = \sqrt{(-1)^2 + 0^2 + 4^2} = \sqrt{1 + 16} = \sqrt{17}\),
            \item \(\|(-4, -17, -1)\| = \sqrt{(-4)^2 + (-17)^2 + (-1)^2} = \sqrt{16 + 289 + 1} = \sqrt{306}\).
          \end{itemize}
          Como os vetores não possuem módulo unitário, o conjunto não é ortonormal.  Como há três vetores em \(\mathbb{R}^3\) e eles são linearmente independentes, o conjunto forma uma base para \(\mathbb{R}^3\).

    \item Os produtos escalares entre os vetores \((2, -4, 2)\), \((0, 2, 4)\) e \((-10, -4, 2)\) são os seguintes:
          \begin{itemize}
            \item \((2, -4, 2) \cdot (0, 2, 4) = 0 - 8 + 8 = 0\),
            \item \((2, -4, 2) \cdot (-10, -4, 2) = -20 + 16 + 4 = 0\),
            \item \((0, 2, 4) \cdot (-10, -4, 2) = 0 - 8 + 8 = 0\).
          \end{itemize}
          O conjunto é ortogonal. O módulo dos vetores é dado por:
          \begin{itemize}
            \item \(\|(2, -4, 2)\| = \sqrt{2^2 + (-4)^2 + 2^2} = \sqrt{4 + 16 + 4} = \sqrt{24}\),
            \item \(\|(0, 2, 4)\| = \sqrt{0^2 + 2^2 + 4^2} = \sqrt{20}\),
            \item \(\|(-10, -4, 2)\| = \sqrt{(-10)^2 + (-4)^2 + 2^2} = \sqrt{100 + 16 + 4} = \sqrt{120}\).
          \end{itemize}
          Como os vetores não possuem módulo unitário, o conjunto não é ortonormal. Como há três vetores em \(\mathbb{R}^3\) e eles são linearmente independentes, o conjunto forma uma base para \(\mathbb{R}^3\).

    \item O primeiro produto escalar é:
          \[
            \left(\frac{\sqrt{2}}{3}, 0, -\frac{\sqrt{2}}{6}\right) \cdot \left(0, \frac{2\sqrt{5}}{5}, -\frac{\sqrt{5}}{5}\right) = 0 + 0 + \frac{\sqrt{2}}{6} \cdot \frac{\sqrt{5}}{5} \neq 0.
          \]
          Logo, o conjunto não é ortogonal.

    \item Os produtos escalares entre os vetores \(\left(\frac{\sqrt{2}}{3}, 0, \frac{\sqrt{2}}{2}\right)\), \(\left(-\frac{\sqrt{6}}{6}, \frac{\sqrt{6}}{3}, \frac{\sqrt{6}}{6}\right)\) e \(\left(\frac{\sqrt{3}}{3}, \frac{\sqrt{3}}{3}, -\frac{\sqrt{3}}{3}\right)\) são os seguintes:
          \begin{itemize}
            \item Primeiro e segundo vetores:
                  \[
                    \left(\frac{\sqrt{2}}{3}, 0, \frac{\sqrt{2}}{2}\right) \cdot \left(-\frac{\sqrt{6}}{6}, \frac{\sqrt{6}}{3}, \frac{\sqrt{6}}{6}\right) = -\frac{\sqrt{12}}{18} + 0 + \frac{\sqrt{12}}{12} = 0.
                  \]
            \item Primeiro e terceiro vetores:
                  \[
                    \left(\frac{\sqrt{2}}{3}, 0, \frac{\sqrt{2}}{2}\right) \cdot \left(\frac{\sqrt{3}}{3}, \frac{\sqrt{3}}{3}, -\frac{\sqrt{3}}{3}\right) = \frac{\sqrt{6}}{9} + 0 - \frac{\sqrt{6}}{6} = 0.
                  \]
            \item Segundo e terceiro vetores:
                  \[
                    \left(-\frac{\sqrt{6}}{6}, \frac{\sqrt{6}}{3}, \frac{\sqrt{6}}{6}\right) \cdot \left(\frac{\sqrt{3}}{3}, \frac{\sqrt{3}}{3}, -\frac{\sqrt{3}}{3}\right) = -\frac{\sqrt{18}}{18} + \frac{\sqrt{18}}{9} - \frac{\sqrt{18}}{18} = 0.
                  \]
          \end{itemize}
          O conjunto é ortogonal. O módulo do primeiro vetor é
          \[
            \sqrt{\left(\frac{\sqrt{2}}{3}\right)^2 + 0^2 + \left(\frac{\sqrt{2}}{2}\right)^2} \neq 1.
          \]
          Portanto, o conjunto não é ortonormal. Como há três vetores em \(\mathbb{R}^3\) e eles são linearmente independentes, o conjunto forma uma base para \(\mathbb{R}^3\).

    \item O produto escalar entre os vetores \((2, 5, -3)\) e \((4, 2, 6)\) é dado por:
          \[
            (2, 5, -3) \cdot (4, 2, 6) = 8 + 10 - 18 = 0.
          \]
          O conjunto é ortogonal. O módulo dos vetores é:
          \[
            \|(2, 5, -3)\| = \sqrt{2^2 + 5^2 + (-3)^2} = \sqrt{4 + 25 + 9} = \sqrt{38} \neq 1.
          \]
          O conjunto não é ortonormal. Como há dois vetores em \(\mathbb{R}^3\), o conjunto não é base para \(\mathbb{R}^3\).

    \item O produto escalar entre \((-6, 3, 2, 1)\) e \((2, 0, 6, 0)\) é dado por:
          \[
            (-6, 3, 2, 1) \cdot (2, 0, 6, 0) = -12 + 0 + 12 + 0 = 0.
          \]
          O conjunto é ortogonal. O módulo dos vetores é:
          \[
            \|(-6, 3, 2, 1)\| = \sqrt{(-6)^2 + 3^2 + 2^2 + 1^2} = \sqrt{36 + 9 + 4 + 1} = \sqrt{50} \neq 1.
          \]
          O conjunto não é ortonormal. Como há dois vetores em \(\mathbb{R}^4\), o conjunto não é base para \(\mathbb{R}^4\).
  \end{enumerate}
\end{resolution}

% =========================== Questão 08 ===========================
\begin{question}
  Aplique o processo de ortogonalização Gram-Schmidt para transformar a base para $\mathbb{R}^{n}$ dada em um base ortonormal. Use os vetores na ordem e que eles são dados.
  \vspace{-12pt}
  \begin{multicols}{2}
    \begin{enumerate}[label=\alph*)]
      \item $B = \{(3, 4), (1, 0)\}$
      \item $B = \{(-1, 2), (1, 0)\}$
      \item $B = \{(0, 1), (2, 5)\}$
      \item $B = \{(4, -3), (3, 2)\}$
      \item $B = \{(2, 1, -2), (1, 2, 2), (2, -2, 1)\}$
      \item $B = \{(1, 0, 0), (1, 1, 1), (1, 1, -1)\}$
      \item $B = \{(4, -3,0), (1, 2, 0), (0, 0, 4)\}$
      \item $B = \{(0, 1, 2), (2, 0, 0), (1, 1, 1)\}$
    \end{enumerate}
  \end{multicols}
\end{question}

% ========================= Resolução 08 =========================
\begin{resolution}
  Por meio do algoritmo imeplementado no trabalho pedido nesta disicplinas, temos:
  \begin{enumerate}[label=\alph*)]
    \item $B = \{(0.6, 0.8), (0.8, -0.6)\}$
    \item $B = \{(-0.45, 0.89), (0.89, 0.45)\}$
    \item $B = \{(0, 1), (1, 0)\}$
    \item $B = \{(0.8, -0.6), (0.6, 0.8)\}$
    \item $B = \{(0.67, 0.33, -0.67), (0.33, 0.67, 0.67), (0.67, -0.67, 0.33)\}$
    \item $B = \{(1, 0, 0), (0, 0.71, 0.71), (0, 0.71, -0.71)\}$
    \item $B = \{(0.8, -0.6, 0.), (0.6, 0.8, 0), (0, 0, 1)\}$
    \item $B = \{(0, 0.45, 0.89), (1, 0, 0), (0, 0.89, -0.45)\}$
  \end{enumerate}
\end{resolution}

% =========================== Questão 09 ===========================
\begin{question}
  Encontre a matriz de coordenadas de $w$ relativo a base ortonormal de $B$ em $\mathbb{R}^{n}$.
  \begin{enumerate}[label=\alph*)]
    \item $w = (1, 2), B = \left\{\left(-\frac{2\sqrt{13}}{13}, \frac{3\sqrt{13}}{13}\right), \left(\frac{3\sqrt{13}}{13}, \frac{2\sqrt{13}}{13}\right)\right\}$.
    \item $w = (4, -3), B = \left\{\left(-\frac{\sqrt{3}}{3}, \frac{\sqrt{6}}{3}\right), \left(-\frac{\sqrt{6}}{3}, \frac{\sqrt{3}}{3}\right)\right\}$.
    \item $w = (2, -2, 1), B = \left\{\left(-\frac{\sqrt{10}}{10}, 0, \frac{3\sqrt{10}}{10}\right), (0, 1, 0), \left(-\frac{3\sqrt{10}}{10}, 0, \frac{\sqrt{10}}{10}\right)\right\}$.
  \end{enumerate}
\end{question}

% ========================= Resolução 09 =========================
\begin{resolution}
  Para encontrar a matriz de coordenadas de \( w \) relativo à base ortonormal \( B \), usamos o fato de que, para bases ortonormais, a coordenada de \( w \) em relação a cada vetor \( v_i \) da base é o produto interno de \( w \) com \( v_i \). Ou seja, se \( B = \{v_1, v_2, \dots, v_n\} \), então:

  \[
    [w]_B = \begin{bmatrix}
      \langle w, v_1 \rangle \\
      \langle w, v_2 \rangle \\
      \vdots                 \\
      \langle w, v_n \rangle
    \end{bmatrix}.
  \]

  O produto interno no \( \mathbb{R}^n \) é dado por:
  \[
    \langle u, v \rangle = \sum_{i=1}^n u_i v_i.
  \]

  \begin{enumerate}[label=\alph*)]
    \item Os vetores da base são:
          \[
            v_1 = \left(-\frac{2\sqrt{13}}{13}, \frac{3\sqrt{13}}{13}\right), \quad v_2 = \left(\frac{3\sqrt{13}}{13}, \frac{2\sqrt{13}}{13}\right).
          \]

          Coordenada relativa a \( v_1 \):
          \[
            \langle w, v_1 \rangle = \left(1\right)\left(-\frac{2\sqrt{13}}{13}\right) + \left(2\right)\left(\frac{3\sqrt{13}}{13}\right) = -\frac{2\sqrt{13}}{13} + \frac{6\sqrt{13}}{13} = \frac{4\sqrt{13}}{13}.
          \]

          Coordenada relativa a \( v_2 \):
          \[
            \langle w, v_2 \rangle = \left(1\right)\left(\frac{3\sqrt{13}}{13}\right) + \left(2\right)\left(\frac{2\sqrt{13}}{13}\right) = \frac{3\sqrt{13}}{13} + \frac{4\sqrt{13}}{13} = \frac{7\sqrt{13}}{13}.
          \]

          Matriz de coordenadas:
          \[
            [w]_B = \begin{bmatrix}
              \frac{4\sqrt{13}}{13} \\[8pt]
              \frac{7\sqrt{13}}{13}
            \end{bmatrix}.
          \]

    \item Os vetores da base são:
          \[
            v_1 = \left(-\frac{\sqrt{3}}{3}, \frac{\sqrt{6}}{3}\right), \quad v_2 = \left(-\frac{\sqrt{6}}{3}, \frac{\sqrt{3}}{3}\right).
          \]

          Coordenada relativa a \( v_1 \):
          \[
            \langle w, v_1 \rangle = \left(4\right)\left(-\frac{\sqrt{3}}{3}\right) + \left(-3\right)\left(\frac{\sqrt{6}}{3}\right) = -\frac{4\sqrt{3}}{3} - \frac{3\sqrt{6}}{3} = -\frac{4\sqrt{3} + 3\sqrt{6}}{3}.
          \]

          Coordenada relativa a \( v_2 \):
          \[
            \langle w, v_2 \rangle = \left(4\right)\left(-\frac{\sqrt{6}}{3}\right) + \left(-3\right)\left(\frac{\sqrt{3}}{3}\right) = -\frac{4\sqrt{6}}{3} - \frac{3\sqrt{3}}{3} = -\frac{4\sqrt{6} + 3\sqrt{3}}{3}.
          \]

          Matriz de coordenadas:
          \[
            [w]_B = \begin{bmatrix}
              -\frac{4\sqrt{3} + 3\sqrt{6}}{3} \\[8pt]
              -\frac{4\sqrt{6} + 3\sqrt{3}}{3}
            \end{bmatrix}.
          \]

    \item Os vetores da base são:
          \[
            v_1 = \left(-\frac{\sqrt{10}}{10}, 0, \frac{3\sqrt{10}}{10}\right), \quad v_2 = (0, 1, 0), \quad v_3 = \left(-\frac{3\sqrt{10}}{10}, 0, \frac{\sqrt{10}}{10}\right).
          \]

          Coordenada relativa a \( v_1 \):
          \[
            \langle w, v_1 \rangle = \left(2\right)\left(-\frac{\sqrt{10}}{10}\right) + \left(-2\right)(0) + \left(1\right)\left(\frac{3\sqrt{10}}{10}\right) = -\frac{2\sqrt{10}}{10} + \frac{3\sqrt{10}}{10} = \frac{\sqrt{10}}{10}.
          \]

          Coordenada relativa a \( v_2 \):
          \[
            \langle w, v_2 \rangle = \left(2\right)(0) + \left(-2\right)(1) + \left(1\right)(0) = -2.
          \]

          Coordenada relativa a \( v_3 \):
          \[
            \langle w, v_3 \rangle = \left(2\right)\left(-\frac{3\sqrt{10}}{10}\right) + \left(-2\right)(0) + \left(1\right)\left(\frac{\sqrt{10}}{10}\right) = -\frac{6\sqrt{10}}{10} + \frac{\sqrt{10}}{10} = -\frac{5\sqrt{10}}{10} = -\frac{\sqrt{10}}{2}.
          \]

          Matriz de coordenadas:
          \[
            [w]_B = \begin{bmatrix}
              \frac{\sqrt{10}}{10} \\[8pt]
              -2                   \\[8pt]
              -\frac{\sqrt{10}}{2}
            \end{bmatrix}.
          \]

  \end{enumerate}
\end{resolution}

% =========================== Questão 10 ===========================
\begin{question}
  Considere o conjunto de polinômios $S = \{1, t, t^2\}$ definida sobre o intervalo $-1 \leq t \leq 1$. Usando o processo de ortogonalização de Gram-Schmidt, obtenha o conjunto orthonormal.
\end{question}

% ========================= Resolução 10 =========================
\begin{resolution}
  Para obter um conjunto ortonormal a partir de \( S = \{1, t, t^2\} \) no intervalo \( -1 \leq t \leq 1 \) usando o processo de Gram-Schmidt, considera-se o produto interno definido como:
  $$ \langle f, g \rangle = \int_{-1}^1 f(t) g(t) \, dt.$$
  Primeiramente, define-se \( q_1 = 1 \), já que o vetor constante é ortogonal por construção. Em seguida, o vetor \( t \) é ortogonalizado por meio da subtração de sua projeção em \( q_1 \), dada por
  $$ \text{proj}_{q_1}(t) = \frac{\langle t, q_1 \rangle}{\langle q_1, q_1 \rangle} q_1.$$
  Calculando \( \langle t, q_1 \rangle = \int_{-1}^1 t \cdot 1 \, dt = 0 \), conclui-se que \( q_2 = t \), pois \( \text{proj}_{q_1}(t) = 0 \).

  Para ortogonalizar \( t^2 \), considera-se
  $$ q_3 = t^2 - \text{proj}_{q_1}(t^2) - \text{proj}_{q_2}(t^2),$$
  com as projeções definidas por:
  $$ \text{proj}_{q_1}(t^2) = \frac{\langle t^2, q_1 \rangle}{\langle q_1, q_1 \rangle} q_1, \quad \text{proj}_{q_2}(t^2) = \frac{\langle t^2, q_2 \rangle}{\langle q_2, q_2 \rangle} q_2.$$
  Calcula-se
  $$ \langle t^2, q_1 \rangle = \int_{-1}^1 t^2 \cdot 1 \, dt = \frac{2}{3},$$
  $$ \langle q_1, q_1 \rangle = \int_{-1}^1 1 \cdot 1 \, dt = 2,$$
  $$ \langle q_2, q_2 \rangle = \int_{-1}^1 t^2 \, dt = \frac{2}{3}$$
  $$ \langle t^2, q_2 \rangle = \int_{-1}^1 t^3 \, dt = 0 .$$
  Assim, \( \text{proj}_{q_1}(t^2) = \frac{\frac{2}{3}}{2} \cdot q_1 = \frac{1}{3} \cdot 1 \), enquanto \( \text{proj}_{q_2}(t^2) = 0 \), resultando em \( q_3 = t^2 - \frac{1}{3} \).

  Para normalizar os vetores, calcula-se suas normas. Para \( q_1 \), obtém-se \( \|q_1\| = \sqrt{\langle q_1, q_1 \rangle} = \sqrt{2} \), levando a \( e_1 = \frac{q_1}{\|q_1\|} = \frac{1}{\sqrt{2}} \). Para \( q_2 \), \( \|q_2\| = \sqrt{\frac{2}{3}} \), resultando em \( e_2 = \frac{q_2}{\|q_2\|} = \sqrt{\frac{3}{2}} t \). Finalmente, para \( q_3 \), calcula-se \( \langle q_3, q_3 \rangle = \int_{-1}^1 \left(t^2 - \frac{1}{3}\right)^2 \, dt = \frac{2}{45} \), o que dá \( \|q_3\| = \sqrt{\frac{2}{45}} \), resultando em \( e_3 = \frac{q_3}{\|q_3\|} = \sqrt{\frac{45}{2}} \left(t^2 - \frac{1}{3}\right) \). Assim, o conjunto ortonormal obtido é:
  $$\left\{\frac{1}{\sqrt{2}}, \sqrt{\frac{3}{2}} t, \sqrt{\frac{45}{2}} \left(t^2 - \frac{1}{3}\right)\right\}.$$
\end{resolution}

% =========================== Questão 11 ===========================
\begin{question}
  Encontre o fatoração QR da matriz abaixo:
  $$\begin{bmatrix}
      1  & 2  & 3 \\
      -1 & -2 & 4 \\
      2  & 0  & 3 \\
      -5 & -3 & 1
    \end{bmatrix}$$
\end{question}

% ========================= Resolução 11 =========================
\begin{resolution}
  Utilizando o algoritmo implementado, obtemos:
  \begin{equation}
    \begin{bmatrix}
      1  & 2  & 3 \\
      -1 & -2 & 4 \\
      2  & 0  & 3 \\
      -5 & -3 & 1
    \end{bmatrix} =
    \begin{bmatrix}
      0.18  & 0.6   & 0.78 \\
      -0.18 & -0.6  & 0.49 \\
      0.36  & -0.53 & 0.34 \\
      -0.9  & 0.03  & 0.19
    \end{bmatrix}
    \begin{bmatrix}
      5.57 & 3.41 & 0.    \\
      0.   & 2.31 & -2.16 \\
      0.   & 0.   & 5.51
    \end{bmatrix}
  \end{equation}
\end{resolution}

% =========================== Questão 12 ===========================
\begin{question}
  Resolva o seguinte conjunto de equações lineares usando a fatorações QR:
  \begin{equation*}
    \begin{bmatrix}
      1 & 2 & 1 \\ -2 & 3 & 4 \\ 1 & 0 & 5
    \end{bmatrix} \begin{bmatrix}
      x_1 \\ x_2 \\ x_3
    \end{bmatrix} =
    \begin{bmatrix}
      6 \\ 20 \\ 14
    \end{bmatrix}
  \end{equation*}
\end{question}

% ========================= Resolução 12 =========================
\begin{resolution}
  Temos que a matriz de transformação pode ser decompostas nas seguintes matrizes:
  $$\begin{bmatrix}
      1 & 2 & 1 \\ -2 & 3 & 4 \\ 1 & 0 & 5
    \end{bmatrix} = \begin{bmatrix}
      0.41  & 0.83 & -0.38 \\
      -0.82 & 0.52 & 0.25  \\
      0.41  & 0.21 & 0.89
    \end{bmatrix}
    \begin{bmatrix}
      2.45 & -1.63 & -0.82 \\
      0.   & 3.21  & 3.94  \\
      0.   & 0.    & 5.08
    \end{bmatrix}$$
  Calculando $y = Q^{\top}b$, temos:
  \begin{equation*}
    y = \begin{bmatrix}
      0.41  & 0.83 & -0.38 \\
      -0.82 & 0.52 & 0.25  \\
      0.41  & 0.21 & 0.89
    \end{bmatrix} ^{\top} \begin{bmatrix}
      6 \\ 20 \\ 14
    \end{bmatrix} = \begin{bmatrix}
      -8.16 \\
      18.25 \\
      15.24
    \end{bmatrix}
  \end{equation*}
  Por fim, calculamos $Rx = y$:
  \begin{equation*}
    \begin{bmatrix}
      2.45 & -1.63 & -0.82 \\
      0.   & 3.21  & 3.94  \\
      0.   & 0.    & 5.08
    \end{bmatrix}
    \begin{bmatrix}
      x_1 \\ x_2 \\ x_3
    \end{bmatrix} = \begin{bmatrix}
      -8.16 \\
      18.25 \\
      15.24
    \end{bmatrix}
  \end{equation*}
  Assim,
  \begin{equation}
    x_1 = -1.00, \quad x_2 = 2.00, \quad x_3 = 3.00
  \end{equation}

\end{resolution}

% =========================== Questão 13 ===========================
\begin{question}
  Resolva o seguinte conjunto de equações lineares usando fatoração QR.
  \begin{align*}
    x_1 + 2 x_2 + 3 x_3 & = 14 \\
    4x_1 + 5x_2 + 6x_3  & = 32 \\
    7x_1 - 3x_2 -2x_3   & = -5
  \end{align*}
\end{question}

% ========================= Resolução 13 =========================
\begin{resolution}
  Temos que a matriz de transformação pode ser decompostas nas seguintes matrizes:
  $$\begin{bmatrix}
      1 & 2 & 3 \\ 4 & 5 & 6 \\ 7 & -3 & -2
    \end{bmatrix} = \begin{bmatrix}
      0.12 & 0.32 & 0.94  \\
      0.49 & 0.8  & -0.34 \\
      0.86 & -0.5 & 0.06
    \end{bmatrix}
    \begin{bmatrix}
      8.12 & 0.12 & 1.6  \\
      0.   & 6.16 & 6.78 \\
      0.   & 0.   & 0.66
    \end{bmatrix}$$
  Calculando $y = Q^{\top}b$, temos:
  \begin{equation*}
    y = \begin{bmatrix}
      0.12 & 0.32 & 0.94  \\
      0.49 & 0.8  & -0.34 \\
      0.86 & -0.5 & 0.06
    \end{bmatrix} ^{\top} \begin{bmatrix}
      14 \\ 32 \\ -5
    \end{bmatrix} = \begin{bmatrix}
      13.17 \\
      32.67 \\
      1.98
    \end{bmatrix}
  \end{equation*}
  Por fim, calculamos $Rx = y$:
  \begin{equation*}
    \begin{bmatrix}
      8.12 & 0.12 & 1.6  \\
      0.   & 6.16 & 6.78 \\
      0.   & 0.   & 0.66
    \end{bmatrix}
    \begin{bmatrix}
      x_1 \\ x_2 \\ x_3
    \end{bmatrix} = \begin{bmatrix}
      13.17 \\
      32.67 \\
      1.98
    \end{bmatrix}
  \end{equation*}
  Assim,
  \begin{equation}
    x_1 = 1.00, \quad x_2 = 2.00, \quad x_3 = 3.00
  \end{equation}
\end{resolution}

